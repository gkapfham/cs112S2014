\documentclass[12pt]{article}
\textwidth = 6.5in
\textheight = 9.05in
\topmargin 0.0in
\oddsidemargin 0.0in
\evensidemargin 0.0in

% set it so that subsubsections have numbers and they
% are displayed in the TOC (maybe hard to read, might want to disable)

\setcounter{secnumdepth}{3}
\setcounter{tocdepth}{3}

\usepackage{epsfig}
\usepackage{listings}
\usepackage{color}

\definecolor{javared}{rgb}{0.6,0,0} % for strings
\definecolor{javagreen}{rgb}{0.25,0.5,0.35} % comments
\definecolor{javapurple}{rgb}{0.5,0,0.35} % keywords
\definecolor{javadocblue}{rgb}{0.25,0.35,0.75} % javadoc

% define widow protection 
        
\def\widow#1{\vskip #1\vbadness10000\penalty-200\vskip-#1}

% define a little section heading that doesn't go with any number

\def\littlesection#1{
\widow{2cm}
\vskip 0.5cm
\noindent{\bf #1}
\vskip 0.1cm
\noindent
}

% A paraphrase mode that makes it easy to see the stuff that shouldn't
% stay in for the final proposal

\newdimen\tmpdim
\long\def\paraphrase#1{{\parskip=0pt\hfil\break
\tmpdim=\hsize\advance\tmpdim by -15pt\noindent%
\hbox to \hsize
{\vrule\hskip 3pt\vrule\hfil\hbox to \tmpdim{\vbox{\hsize=\tmpdim
\def\par{\leavevmode\endgraf}
\obeyspaces \obeylines 
\let\par=\endgraf
\bf #1}}}}}

\renewcommand{\baselinestretch}{1.2}    % must go before the begin of doc

% go with the way that CC sets the margins

\pagestyle{empty}

\begin{document}

\lstset{language=Java,
basicstyle=\ttfamily,
keywordstyle=\color{javapurple}\bfseries,
stringstyle=\color{javared},
commentstyle=\color{javagreen},
morecomment=[s][\color{javadocblue}]{/**}{*/},
%numbers=left,
numberstyle=\scriptsize\color{black},
stepnumber=1,
numbersep=7pt,
tabsize=4,
showspaces=false,
showstringspaces=false}

% handle widows appropriately
\def\widow#1{\vskip #1\vbadness10000\penalty-200\vskip-#1}

\begin{center}

CS 112: Introduction to Computer Science II \\
Examination 1 \\
%Friday, November 2, 2007 \\

\end{center}

\noindent Answer the five questions that are listed on the following sheets.  You must provide answers to these
questions on a separate sheet of paper.  Please develop responses that clearly express your ideas in the most succinct
manner possible.  You are not permitted to complete this examination in conjunction with any of your classmates.
Furthermore, you cannot consult any outside references during this examination.  If you have questions concerning the
problems, then please visit my office during the examination period.  If you leave the classroom to take the exam, then
you are responsible for checking the white board for status updates.

\newpage

\begin{enumerate}

\item ({\bf 10 Points}) Arithmetic expressions can be used inside of conditional logic statements and iteration
  constructs.  Answer the following questions about arithmetic, variables, and the basic use of the tools that support
  programming in the Java language.

\begin{enumerate}

\item ({\bf 4 Points}) The following code segment uses arithmetic expressions to change the value of {\tt int}
  variables.  Show exactly what is output by the following sequence of Java statements (you may assume that these
  statements are contained within the {\tt main} method of an {\tt ArithmeticExpression} class):

  \begin{verbatim}
        final int TRUE = 1;
        int a = 10;
        int b = 20;
        int c = TRUE;
        int d = TRUE;
        
        a = a + b;
        b = a - b;
        a = a - b;
        c = a % 2;
        d = b % 3;
        
        System.out.println(" a = " + a + " , b = " + b);
        System.out.println(" c = " + c + " , d = " + d);
  \end{verbatim}
  \vspace*{-.2in}

\item ({\bf 4 Points}) What are the input(s), output(s), and purpose(s) of a Java compiler and virtual machine?  You
  should answer this question by drawing an easy-to-read diagram that contains meaningful labels, arrows, and boxes.

%% \item ({\bf 2 Points}) Discuss the similarities and differences
%%   between {\em instance} variables and {\em static} variables.  Also,
%%   please explain the similarities and differences between the keywords
%%   {\tt private} and {\tt protected}.

%% \item ({\bf 2 Points}) The Java programming language provides both the
%%   {\tt int} and the {\tt Integer} data types.  Explain how the Java
%%   virtual machine passes parameters when a method is called.  Why
%%   can't we write a {\tt swap(int a, int b)} or {\tt swap(Integer a,
%%     Integer a)} method?

%% \item ({\bf 2 Points}) What is an array?  The Java programming
%%   language provides facilities for {\em array bounds checking}.
%%   Please describe what array bounds checking is and discuss the
%%   strengths and weaknesses of this technique.

% \item ({\bf 3 Points}) The Java programming language uses the {\tt
%   CLASSPATH} environment variable.  Explain how the Java compiler and
%   virtual machine use this environment variable during the compilation
%   and execution of a Java program.
% 

\item ({\bf 2 Points}) Suppose that a Java program contained the
  following variable declaration: {\tt ArrayList<Tweet> validTweets =
    new ArrayList<Tweet>();}.  What is an {\tt Arraylist}?  What does
  it mean when the {\tt ArrayList} called {\tt validTweets} is
  declared with the syntax {\tt ArrayList<Tweet>}?

\end{enumerate}

\newpage

\item ({\bf 10 Points}) The Java programming language supports many
  different primitive data types and constructs for solving problems.
  Answer the following questions about these features of the Java
  programming language.

\begin{enumerate}

\item ({\bf 3 Points}) The Java programming language supports the
  definition of methods that have a {\tt void} return type.  What does
  it mean if a method returns {\tt void}?  Your response should give
  an example of a method that could reasonably have a {\tt void}
  return type.  Please explain why your chosen method should be {\tt
    void}.

%% \item ({\bf 3 Points}) Briefly define the terms {\em inheritance},
%%   {\em encapsulation}, and {\em polymorphism}.  For the terms
%%   inheritance and encapsulation, please ensure that your response
%%   states how the Java programming language provides these features.
  
\item ({\bf 4 Points}) The Java programming language provides many data types such as {\tt int}, {\tt double}, {\tt
  long}, and {\tt boolean}.  Provide a description of each of these primitive data types.  Your response should include
  a statement of the bit-depth of each type. You should also compare and contrast the types whenever possible.
  

%% \item ({\bf 2 Points}) The Java programming language allows a
%%   programmer to declare both {\em instance} and {\em static}
%%   variables.  After showing how to declare these variables in a Java
%%   program, your answer to this question should furnish a clear
%%   definition of these two terms and a discussion of when you would use
%%   them in a Java program.

\item ({\bf 3 Points}) The Java programming language provides several different types of iteration constructs.  After
  explaining the purpose of an iteration construct, please compare and contrast the {\tt while}, {\tt do-while}, and
  {\tt for} loops.

\end{enumerate}

\newpage

\item ({\bf 10 Points}) The Java programming language is an
  object-oriented programming language that supports recursion and
  iteration.  Answer the following questions about the use of
  recursion and iteration in the Java language.

\begin{enumerate}
  
\item ({\bf 3 Points}) Draw a call tree to explain the recursive calls that would occur when {\tt recursiveFactorial(4)}
  is executed.  Make sure that your diagram shows the input parameter and return value for each recursive call in a
  format similar to those that we created during class discussions.  Your response can assume that the {\tt
  recursiveFactorial(4)} call occurs inside of the {\tt main(String[] args)} method.

\begin{verbatim}
public static int recursiveFactorial(int n)
{
  if(n == 0)
     return 1;
  else
  {
     int result = n * recursiveFactorial(n - 1);
     return result;
  }
} 
\end{verbatim}

\item ({\bf 2 Points}) Suppose that the above method is provided by a
  class called {\tt Recursive}.  Next, assume that the {\tt Recursive}
  class has a {\tt main} that calls {\tt recursiveFactorial}.  Explain
  what would happen if a call to {\tt recursiveFactorial(-100)} occurs
  inside of the {\tt main} method within the {\tt Recursive} class.

%% \item ({\bf 2 Points}) Explain how the Java virtual machine uses a
%%   program stack in order to support the execution of recursive
%%   methods.  Provide a picture of the program stack before the
%%   invocation of {\tt factorial(4)}.  Next, draw a picture of the
%%   program stack after the {\tt factorial} method reaches the base
%%   case.

\item ({\bf 5 Points}) Provide the source code for the method called {\tt iterativeFactorial}.  Using the same input
  parameter (i.e., {\tt int n}), your method must compute the same output as the {\tt recursiveFactorial} when given the
  same input value.  However, instead of using a recursive approach, you must compute the factorial value in an
  iterative manner.  (Hint: consider using either a {\tt while} or a {\tt for} loop).

%% \item ({\bf 2 Points}) The Java virtual machine (JVM) uses both a
%%   stack and a heap.  Please clearly explain how the JVM uses these
%%   data structures to support the correct execution of a Java program.
%%   Your response should specifically mention how one or both of these
%%   structures enable the execution of a recursive method.

\end{enumerate}

\newpage

%% Software testing is one technique that we can use to establish a
%% confidence in the correctness of a program.  Goodrich and Tamassia
%% introduce the concept of running time and then examine the usage of
%% empirical and analytical studies of program running time.

\item ({\bf 10 Points}) The implementation of an algorithm must be
  {\em correct} and it must exhibit acceptable {\em performance}.
  Answer the following questions about program testing and debugging,
  using the source code in Figure~\ref{Kinetic} for part (a) of this
  question.

\begin{enumerate}
  
\item ({\bf 5 Points}) In light of the fact that $K =
  \frac{1}{2}mv^2$, it is clear that the {\tt computeVelocity} method
  has a defect in it!  What is the defect in this method?  Please
  provide inputs for {\tt kinetic} and {\tt mass} that will reveal
  this defect.  Moreover, you should give an example input that will
  not identify the fault in {\tt computeVelocity}.  Your response
  should clearly indicate why your chosen inputs will expose the
  defect.

\item ({\bf 5 Points}) The {\tt Weeee} class has a {\tt main} method
  that calls the recursive method {\tt weeee}.  What output will be
  produced when this program executes in a terminal window?  Your
  response to this question should clearly explain why the program
  will behave the way that you have specified.

\begin{verbatim}
public class Weeee
{

    public static void weeee()
    {

        System.out.println(" Weeee! ");
        weeee();

    }

    public static void main(String[] args)
    {

        weeee();

    }

}
\end{verbatim}


%% \item ({\bf 3 Points}) Give an example of three separate methods that
%%   have a worst-case time complexity of $O(n)$, $O(n^2)$, and $O(n^3)$
%%   respectively.  The first method with $O(n)$ time complexity should
%%   be called {\tt methodOne}, the second method can be called {\tt
%%     methodTwo}, and the third is {\tt methodThree}.  {\em As long as
%%     the method has the correct worst-case time complexity, you may
%%     write any valid Java code}.

%% \item ({\bf 3 Points}) Equation~(\ref{eq:sum}) provides a formula that
%%   can be used to calculate summations.  Give a method called {\tt
%%     iterativeSum(int n)} that uses a {\tt for} or {\tt while} loop to
%%   sum the numbers from 1 to $n$.  Next, please implement a {\tt
%%     computeSum(int n)} method that uses Equation~(\ref{eq:sum}) to
%%   directly perform the summation.  Which method is faster?  Or, do the
%%   methods exhibit the same run-time performance?  Regardless of your
%%   view about the performance of these algorithms, your response must
%%   include a complete justification of your response.

%%   \begin{equation}
%%     \sum_{i=1}^{n}i = \frac{n \times (n+1)}{2}
%%     \label{eq:sum}
%%   \end{equation}

%% \item ({\bf 2 Points}) Explain the function $g(n)$ that defines the
%%   complexity class $O(g(n))$ for the function $f(n) = 2^{100}$ and
%%   $f(n) \in O(g(n))$.  You should defend why you selected the $g(n)$
%%   that you did.

%% \item ({\bf )

%% \item ({\bf 3 Point}) Which function grows the {\em fastest}? Which
%%   function grows the {\em slowest}?  {\em Why}?  Your response should
%%   include a mathematical justification.

%%   \begin{enumerate}

%%   \item $f(n) = n^2$

%%   \item $f(n) = n$

%%   \item $f(n) = n!$

%%   \item $f(n) = n^n$

%%   \item $f(n) = 2^n$

%%   \item $f(n) = \frac{n}{n}$

%%   \end{enumerate}

\end{enumerate}

\begin{figure}[p]

\begin{verbatim}
import java.lang.Math;
public class Kinetic
{
  public static String computeVelocity(int kinetic, int mass)
  {
    int velocity_squared = 0;
    int velocity = 0;
    StringBuffer final_velocity = new StringBuffer();
    if( mass != 0 )
    {
      velocity_squared = 3 * (kinetic / mass);
      velocity = (int)Math.sqrt(velocity_squared);
      final_velocity.append(velocity);
    }
    else
    {
      final_velocity.append("Undefined");
    }
    return final_velocity.toString();
  }
}
\end{verbatim}

\caption{The {\tt Kinetic} class with a {\tt computeVelocity} method.}
\label{Kinetic}

\end{figure}

\newpage

\item ({\bf 10 Points}) The Java programming language supports the use
  of a wide variety of primitive and reference data types, including
  arrays.  Answer the following questions about the use of variables
  in Java programs.

\begin{enumerate}

\item ({\bf 4 Points}) Support that a Java program contained the
  following line of code in a {\tt main} method: {\tt int[] six = new
    int[6];}.  Would this line of code cause the compiler to produce
  an error?  After answering this first question, draw a picture of
  what this array would look like in the memory of the Java virtual
  machine.  Finally, you should furnish one line of array-accessing
  code that would cause the program to throw an exception and one that
  would not throw an exception.

\item ({\bf 4 Points}) The Java programming language allows you to
  define inheritance hierarchies that show the relationship between
  objects.  Using proper labels and real Java class names, draw a
  diagram that illustrates an inheritance hierarchy with one {\em
    superclass} and at least one {\em subclass}.  After you have
  finished your diagram, please explain why you designed the
  inheritance hierarchy in this way.

% \item ({\bf 2 Points}) Suppose that a Java program contained the
%   following variable declaration: {\tt ArrayList<Tweet> validTweets =
%     new ArrayList<Tweet>();}.  What is an {\tt Arraylist}?  What does
%   it mean when the {\tt ArrayList} called {\tt validTweets} is
%   declared with the syntax {\tt ArrayList<Tweet>}?
% 
\end{enumerate}


\end{enumerate}

\end{document}



