\documentclass[12pt]{article}             
\textwidth = 6.5in
\textheight = 9.05in
\topmargin 0.0in
\oddsidemargin 0.0in
\evensidemargin 0.0in

% set it so that subsubsections have numbers and they
% are displayed in the TOC (maybe hard to read, might want to disable)

\setcounter{secnumdepth}{3}
\setcounter{tocdepth}{3}

% define widow protection 
        
\def\widow#1{\vskip #1\vbadness10000\penalty-200\vskip-#1}

% define a little section heading that doesn't go with any number

\def\littlesection#1{
\widow{2cm}
\vskip 0.5cm
\noindent{\bf #1}
\vskip 0.1cm
\noindent
}

% A paraphrase mode that makes it easy to see the stuff that shouldn't
% stay in for the final proposal

\newdimen\tmpdim
\long\def\paraphrase#1{{\parskip=0pt\hfil\break
\tmpdim=\hsize\advance\tmpdim by -15pt\noindent%
\hbox to \hsize
{\vrule\hskip 3pt\vrule\hfil\hbox to \tmpdim{\vbox{\hsize=\tmpdim
\def\par{\leavevmode\endgraf}
\obeyspaces \obeylines 
\let\par=\endgraf
\bf #1}}}}}

\renewcommand{\baselinestretch}{1.2}    % must go before the begin of doc

% go with the way that CC sets the margins

\vspace*{-1in}

\begin{document}

% handle widows appropriately
\def\widow#1{\vskip #1\vbadness10000\penalty-200\vskip-#1}

\begin{center}

CS 112: Introduction to Computer Science II \\
Examination Two \\
%Monday, November 20, 2006 \\

\end{center}

\noindent
Answer the five questions that are listed below.  You must provide
answers to these questions on a separate sheet of paper.  Please
develop responses that clearly express your ideas in the most succinct
manner possible.  You are not permitted to complete this examination
in conjunction with any of your classmates.  Furthermore, you cannot
consult any outside references during this examination.  If you have
questions concerning the problems that are listed below, please visit
my office during the examination period.  If you leave the classroom
to take the exam, you are responsible for checking the white board for
status updates.

\begin{enumerate}
  
\item ({\bf 10 Points}) The {\tt Queue} is an abstract data type that
  adheres to the ``first in, first out'' (FIFO) discipline.  Answer
  the following questions about {\tt Queue}s.

  \begin{enumerate}
    
%%     \item ({\bf 2 Points}) Suppose that $N$ denotes the size of the
%%       array that is used to implement a {\tt Queue}.  Furthermore,
%%       assume that $f$ represents the {\tt int} pointer to the front of
%%       the {\tt Queue} and $r$ is the {\tt int} pointer to the rear of
%%       the {\tt Queue}.  What is the equation to calculate the size of
%%       the {\tt Queue} that uses a circular array?  Your response
%%       should include the general equation, example values for the
%%       three variables, and a demonstration of how to calculate the {\tt
%%       Queue} size for these variables.

%%     \item ({\bf 6 Points}) It is possible to implement a {\tt Queue}
%%       that contains an array that is accessed in a circular fashion.
%%       In your response this question, you may assume that an instance
%%       of {\tt ArrayQueue} contains an internal array {\tt storage[]}
%%       of size three.  For each of the following {\tt Queue}
%%       operations, show the state of {\tt storage[]} and the values for
%%       {\tt f}, the ``front'' pointer, and {\tt r}, the ``rear''
%%       pointer.

%%     \begin{verbatim}
%%        1. enqueue(100), 2. dequeue(), 
%%        3. enqueue(200), 4. dequeue(), 
%%        5. enqueue(300), 6. dequeue()
%%     \end{verbatim}

    %% \item ({\bf 2 Points}) It is possible to implement the {\tt Queue}
    %%   data structure with a circular array.  If {\tt dequeue} updates
    %%   the $f$ pointer and {\tt enqueue} changes the $r$ pointer, then
    %%   what assignment statements do these operations use to modify
    %%   their pointers?

    \item ({\bf 5 Points}) Furnish an implementation of a complete
      Java program that accepts an arbitrary number of command-line
      arguments and then places the arguments into a {\tt Queue}.
      Assuming that the {\tt enqueue} of a single argument is the
      basic operation, what is the worst-case time complexity of this
      method?  Your response to this question should include a
      justification for your chosen time complexity.

    \item ({\bf 5 Points}) Draw a table that has three columns.  Label
      the first column ``Operation'', the second column ``Output'',
      and the third column ``front $\leftarrow Q \leftarrow$ rear''.
      Populate the second and third columns with the output and {\tt
        Queue} state, respectively, that are created by the following
      sequence of operations.  You can place one operation on a single
      row of the table.

      \begin{verbatim} 
	1. enqueue(55), 2. enqueue(13), 
	3. dequeue(),   4. enqueue(79),
	5. dequeue(),   6. front(), 
	7. dequeue(),   8. dequeue() 
      \end{verbatim}

    \end{enumerate}
      
\newpage

%% The extendable array data structure uses an underlying array that can
%% grow when its space is exhausted.

\item ({\bf 10 Points}) The linked list is a data structure that
  enables other structures to grow in a dynamic fashion.  However, it
  is also possible to implement a {\tt Stack} and a {\tt Queue} with
  an array.  Answer the following questions about the trade-offs
  associated with using linked lists and arrays to implement data
  structures.

  \begin{enumerate}

%%   \item ({\bf 5 Points}) Clearly explain what happens when an
%%     extendable array is full and then the {\em add(i,e)} method is
%%     invoked. You should discuss the four steps that must be taken in
%%     order to properly handle this type of situation.  Your response
%%     must also explain how the garbage collector (GC) within the Java
%%     virtual machine (JVM) is used during memory management.

%%   \item ({\bf 6 Points}) Identify the similarities and differences
%%     between the {\tt java.util.Vector} and the {\tt
%%       java.util.ArrayList}.  Your response should explain similarities
%%     and differences related to both the performance and the
%%     functionality of these implementations of the extendable array
%%     abstract data type (ADT).

  \item ({\bf 2 Points}) Clearly explain the structure of the {\tt
    Node} and the {\tt DNode}.  Your response should indicate which of
    these types is used for the singly linked list and which is used
    for the doubly linked list.  Finally, your answer should include a
    simple graphical depiction of the {\tt Node} and the {\tt DNode}.

%%   \item ({\bf 2 Points}) Two important data structures include the
%%     linked list (LL) and the doubly linked list (DLL).  Please furnish
%%     a clearly labelled diagram that explains the structure of the node
%%     that makes up an LL and a DLL.

  \item ({\bf 4 Points}) It is possible to implement a stack with
    either a linked list or a fixed size array.  What are the
    trade-offs associated with the array-based and linked list-based
    approaches to implementing a stack?  In your response to this
    question, you may use the abbreviation ``AS'' to refer to the
    array-based stack and ``LLS'' to stand for the stack that uses a
    linked list as its storage mechanism.

  \item ({\bf 4 Points}) Suppose that you have a properly implemented
    array-based queue and linked list-based queue.  What is the
    worst-case time complexity of the {\tt enqueue} and {\tt dequeue}
    methods for both of these queue implementations?  Your response to
    this question should include a clear justification for your chosen
    complexities.

%%   \item ({\bf 4 Points}) Sketch the implementation of either the {\em
%%     add(i,e)} or {\em remove(i)} method that is part of a simple
%%     array-based implementation of an extendable array.  Your sketch
%%     should clearly explain how to perform the shifting of array
%%     elements.  The parameter $i$ denotes the index subject to addition
%%     or removal and the parameter $e$ represents the data element that
%%     must be added.

%%   \item ({\bf 2 Points}) Suppose that we were interested in storing an
%%     instance of an extendable array on the file system.  Explain how
%%     we could use a Java serialization primitive to perform this
%%     operation.  What is the difference between the XStream serialization
%%     primitive and the traditional Java serialization primitive?

  \end{enumerate}

\newpage

\item ({\bf 10 Points}) The {\tt Stack} is an abstract data type (ADT)
  that adheres to the ``last in first out'' (LIFO) discipline.  Answer
  the following questions about the {\tt Stack} ADT.

  \begin{enumerate}

  \item ({\bf 5 Points}) Provide a Java-based pseudo-code
    implementation (or, just plain pseudo-code implementation) of the
    {\tt push} and {\tt pop} methods for a simple array-based
    implementation of the {\tt Stack}.  You can assume that the {\tt
      Stack} is implemented with a fixed size array called {\tt S} and
    the top of the stack pointer is stored in an {\tt int} variable
    called {\tt t}.

  \item ({\bf 5 Points}) The {\tt Stack} abstract data type provides
    the methods such as {\tt isEmpty}, {\tt top}, {\tt push}, and {\tt
      pop}.  What is the worst-case time complexity for each of these
    methods?  Why did you pick these complexities? If we assume that
    an $N$ element array called {\tt S} is used to store the elements
    in the {\tt Stack}, what is the worst-case space complexity of the
    array-based {\tt Stack} implementation? Why?

  %% \item ({\bf 2 Points}) Using one or more diagrams and a textual
  %%   explanation, please demonstrate how to use a {\tt Stack} to
  %%   implement the ``simplified back button'' functionality as found in
  %%   a Web browser such as Lynx.

  \end{enumerate}

\newpage

\item ({\bf 10 Points}) Goodrich and Tamassia claim that the {\tt
  Stack} ADT is the ``simplest of all data structures.''  Answer the
  following questions about the {\tt Stack}.

\begin{enumerate}

\item ({\bf 4 Points}) Draw a table that has three columns.  Label the
  first column ``Operation'', the second column ``Output'', and the
  third column ``Stack Contents''.  Populate the second and third
  columns with the output and {\tt Stack} state, respectively, that
  are created by the following sequence of operations.  You can place
  one operation on a single row.  Make sure that you clearly indicate
  where the top of the stack is in your column that describes the
  state of the {\tt Stack}.

  \begin{verbatim} 
    1. push(15),    2. push(32), 
    3. pop(),       4. push(79),
    5. pop(),       6. top(), 
    7. pop(),       8. pop() 
  \end{verbatim}

  \vspace*{-.3in}

\item ({\bf 2 Points}) The Java virtual machine (JVM) uses both a
  stack and a heap.  Please clearly explain how the JVM uses these
  data structures to support the correct execution of a Java program.

%% \item ({\bf 2 Points}) The time complexity of the {\tt LinkedList}'s
%%   {\em get(i)} method is given as $O(min(i+1, n-i))$.  Clearly explain
%%   the meaning of this time complexity.  Your response to this question
%%   should state the meaning of the variables $i$ and $n$.  Finally,
%%   please identify the worst-case input parameter for the {\em get}
%%   method.  Your response should include a graphical depiction of an
%%   instance of the {\tt LinkedList} class.

\item ({\bf 4 Points}) Explain which methods of the array-based {\tt
  Stack} abstract data type can throw the {\tt EmptyStackException}
  and the {\tt FullStackException}.  Your response to this question
  should discuss why your selected methods can throw these exceptions.
  Your answer should also include a graphical depiction of a {\tt
    Stack} for each circumstance in which an exception can be thrown.

\end{enumerate}

\newpage

\item ({\bf 10 Points}) The {\tt Queue} is important abstract data
  types (ADT) that is used in many real-world software applications
  like operating systems.  Answer the following questions about the
  design, implementation, and evaluation of this abstract data type.

\begin{enumerate}

%%   \item ({\bf 2 Points}) Clearly explain the structure of the {\tt
%%     Node} and the {\tt DNode}.  Your response should indicate which of
%%     these types is used for the singly linked list and which is used
%%     for the doubly linked list.  Finally, your answer should include a
%%     simple graphical depiction of the {\tt Node} and the {\tt DNode}.

%% \item ({\bf 4 Points}) Completely describe the operation of the {\tt
%% addFirst}, {\tt addLast}, {\tt removeFirst}, and {\tt removeLast}
%% operations for a {\tt DoubleEndedQueue} that is implemented with a 
%% doubly linked list.  Your response should explain when (and if!) any 
%% of these methods could throw an exception.

%%   \item ({\bf 2 Points}) The {\tt java.util.Vector} has a constructor
%%     that accepts two parameters in the following manner: {\tt
%%       Vector(int initialCapacity, int capacityIncrement)}.  Explain
%%     the purpose of each parameter and discuss how it will impact the
%%     performance of the {\tt Vector} abstract data type.

%% \item ({\bf 3 Points}) What is a {\tt java.util.Iterator}?  Write a
%%   short code segment to demonstrate the use of an {\tt Iterator}.
%%   Your response should explain how an {\tt Iterator} is extracted from
%%   a {\tt java.util.ArrayList} and fully describe the methods furnished
%%   by an instance of {\tt Iterator}.  As part of your response to this
%%   question, you can assume that an {\tt ArrayList} called {\tt
%%     number\_list} has already been populated with subclasses of {\tt
%%     java.lang.Number}.

%%   \item ({\bf 4 Points}) Many programming languages, such as Java,
%%     provide data structures such as arrays and the {\tt ArrayList}.
%%     In your response to this question, please explain the similarities
%%     and differences between an array and an {\tt ArrayList}.  Whenever
%%     possible, your answer should include a properly labeled diagram to
%%     illustrate each of your key points.

  %% \item ({\bf 6 Points}) It is possible to implement a {\tt Queue}
  %%   abstract data type with a {\em singly} {\tt LinkedList}.  Please
  %%   furnish two diagrams that separately describe an {\em efficient}
  %%   and {\em inefficient} implementation of a {\tt Queue} with a {\tt
  %%     LinkedList}.  Your diagrams should clearly mark the front and
  %%   rear of the {\tt Queue} and the positioning of the {\tt
  %%     LinkedList} inside of the {\tt Queue}.  Finally, you must
  %%   comment on why each of the implementations is either efficient or
  %%   inefficient.

  %% \item ({\bf 2 Points}) What is a compression algorithm and when
  %%   would you use it?  Next, suppose that you decided to input a
  %%   populated instance of the {\tt ArrayBlockingQueue} class into a
  %%   compression technique.  If $L_B$ is the size of the {\tt
  %%     ArrayBlockingQueue} {\em before} compression and $L_A$ is the
  %%   size {\em after} compression, then how would you calculate the
  %%   {\em percent reduction} (PR) of the {\tt ArrayBlockingQueue}?
  %%   Your response to the second part of this question should include
  %%   an equation that the defines the percent reduction.  What does it
  %%   mean if PR is positive or negative?

\item ({\bf 2 Points}) A complete implementation of an {\tt
  ArrayQueue} requires the creation and use of the following
  interface:
  
  \vspace*{-.1in}

  \begin{verbatim}
     public interface Queue<E> 
  \end{verbatim}

  \vspace*{-.3in}

  \noindent
  Answer the following questions about this {\tt interface}
  declaration:

  \begin{enumerate}

    \item What is the meaning of the term {\tt interface}?

    \item What is the meaning of the notation {\tt Queue<E>}?
    
  \end{enumerate}

\item ({\bf 4 Points}) Suppose that you are implementing a {\tt Queue}
  with an array called {\tt Q}.  In this implementation, {\tt Q[0]}
  always denotes the front of the {\tt Queue} and the structure is
  allowed to grow from there.  Is this an efficient design? Why or why
  not?

\item ({\bf 4 Points}) The double-ended queue, or deque, is a separate
  data structure from the queue.  What are the similarities and
  differences between the deque and the queue?  Your response to this
  question should include details about the names and behavior of the
  methods that these structures provide.

%% \item ({\bf 2 Points}) The worst-case time complexity of the {\em
%%   add(i,e)} method is given as $O(n-i+1)$ where there are $n$ elements
%%   currently stored in an array of size $N$.  Clearly explain the
%%   meaning of this time complexity and identify the worst-case values
%%   for the input parameters.

\end{enumerate}

\end{enumerate}

\end{document}



