%!TEX root=cs112S2014-lab2.tex
% mainfile: cs112S2014-lab2.tex 

\input{labspre.tex}

\usepackage[compact]{titlesec}

\begin{document}
\MYTITLE{Laboratory Assignment One: Using Vim as an Integrated Development Environment}
\MYHEADERS{Laboratory Assignment One}{Due: February 4, 2014}

\section*{Introduction}

Practicing software developers normally use an integrated development environment (IDE) to manage various tasks associated with the
design, implementation, and testing of data structures and algorithms. In this course, we will use Vim as an IDE.  In this laboratory
assignment, you will work with your team members to learn about the basic features associated with Vim and then individually
prepare your own tutorial that explains how to use Vim runtime configuration files and plugins to navigate the source code of a
Java program.  

\section*{Using Runtime Configuration Files}

It is very easy to configure Vim by writing VimScript in your {\tt .vimrc} and {\tt .gvimrc} files.  To complete this phase of the
assignment you should run the {\tt git pull} command inside of your {\tt cs112S2014-share} Git repository.  Now, change into the
{\tt labs/lab2/src/dotfiles/} directory.  What files did you find in this location? Now, you should copy these files from the git
repository into the root of your home directory. At this point, you should run the GVim command so that you can see the source
code of one of the Java programs that we used during our class sessions. Do you see that the color scheme is different? If you
would like, you can customize the color scheme by using the ``Edit'' and ``Color Scheme'' menus.  Finally, you should also use
GVim to study the VimScript in the {\tt .vimrc} and {\tt .gvimrc} files.  Make sure that you and your team members understand
these configuration files. What new features does GVim now have?

% After saving this file to the root of your home directory, you should decompress it using the {\tt tar} command in your terminal
% window.  Now, please restart Vim.  What other features have now been added to Vim?  To learn more about how I have configured Vim
% for use in Computer Science 290 Fall 2013, you should study the VimScript in the {\tt .vimrc} and {\tt .gvimrc} files.  Make sure
% that you and your team members understand these configuration files.

\section*{Summary of the Required Deliverables}

  This assignment invites you to submit one printed version of the following deliverables:

  \vspace*{-.05in}
  \begin{enumerate}
	
    \item A description of the steps that a user must take to configure Git and Bitbucket

    \item A description of the inputs, outputs, and behavior of the six aforementioned Git commands

    \item A commented version of the {\tt Hooray.java} and {\tt Weeee.java} programs
	
    \item A one-page report that clearly responds to the following four prompts:
  
      \vspace*{-.05in}
      \begin{enumerate}

	\item The steps that you took to compile and run both of these programs

	\item The output that each of these programs produce

	\item An explanation for why these programs create the output that they do

	\item A discussion of the similarities and differences between these programs and their output

      \end{enumerate}

  \end{enumerate}

Before you turn in this assignment, you also must ensure that the course instructor has read access to your Bitbucket
repository that is named according to the convention {\tt cs112S2014-<your user name>}. Please note that each student in
the class is responsible for completing and submitting their own version of this assignment.  However, you also will be
assigned to work to a team that is tasked with ensuring that all of its members are able to complete each step of the
assignment.  Team members should make themselves available to each other to answer questions and resolve any problems
that develop during the laboratory session. While it is acceptable for members of a team to have high-level
conversations, you should not share source code or full command lines with your team members. To ensure that you can
communicate effectively, members of each team should sit next to each other in the room.  Please see the instructor if
you have questions about this policy.



\end{document}
