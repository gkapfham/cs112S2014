%!TEX root=cs112S2014-lab4.tex
% mainfile: cs112S2014-lab4.tex 

\input{labspre.tex}

\usepackage[compact]{titlesec}

\begin{document}
\MYTITLE{Laboratory Assignment Four: Implementing and Evaluating a Twitter Client}
\MYHEADERS{Laboratory Assignment Four}{Due: February 18, 2014}

\section*{Introduction}

As we continue to both refamiliarize ourselves with the fundamental characteristics of the object-oriented programming paradigm
and to practice using the some of the features of the Vim integrated development environment (IDE), we will take the final steps
towards implementing a complete Twitter client.  Along with learning how to use, enhance, and extend a Java class, we will also
review the steps that you must take to perform console input and output. This laboratory assignment represents the second of a
two-part series of assignments designed to review Java programming.

\section*{Accessing and Understanding GatorTweet}

To start this laboratory assignment, you should return to the {\tt cs112S2014-share} Git repository and type the command {\tt git
  pull} in the terminal window.  Now, you should have a {\tt lab4/GatorTweet/} directory that you can explore further.  Where is
the Java source code in this new directory? Why is it stored in the directory where you found it? Now, you should use GVim to
study the source code in the {\tt build.xml} file.  As you did for the past assignment, you can type the command {\tt :make} in
your GVim window and it will compile the three Java classes and save the bytecode in the correct subdirectories inside of the
{\tt bin/} directory. 

\section*{Connecting to the Twitter Service}



\section*{Finishing the GatorTweet System}

You will notice that the {\tt Tweet} class contains two instance variables and several methods.  First, you should add proper
JavaDoc comments to all of the variables and methods in {\tt Tweet.java}.  Once you have finished adding in all of the necessary
comments, please notice that the {\tt isValidMessage} method is not correctly implemented.  You should enhance this method so that
it returns {\tt true} when a candidate message is valid and {\tt false} otherwise.  This method should classify a message as valid
as long as its length is greater than zero and less than or equal to 140.

% \section*{Adding a GatorTweet Class}

Now, you should implement a {\tt GatorTweet} class that performs the following operations in {\tt main}:

\vspace*{-.1in}
\begin{enumerate}
  % \itemsep 0in
  \item Constructs an {\tt ArrayList} that will only hold {\tt Tweet} objects for the valid {\tt Tweet}s.
  
  \item Constructs an {\tt ArrayList} that will only hold {\tt String}s for the invalid tweet messages.
  
  \item Uses the {\tt java.util.Scanner} call to read in consecutive lines of text from a file in the {\tt tweets/} subdirectory of
    the {\tt GatorTweet/} directory, called ``Tweets.txt'', that contains potential tweets. (It is worth noting that this file can
    contain both valid and invalid tweets).
  
  \item Checks to see if the current tweet message is a valid one or not.  If the message is valid, then it should construct a new
    {\tt Tweet} and store it inside of the {\tt ArrayList} for the valid instances of the {\tt Tweet} class.  If the message is
    invalid, then it should not construct a {\tt Tweet} and instead store the {\tt String} message in the {\tt ArrayList} for
    invalid tweet messages.

  \item Outputs all of the {\tt Tweet}s stored in the {\tt ArrayList} for the valid tweet messages. 

  \item Outputs all of the {\tt Tweet}s stored in the {\tt ArrayList} for the invalid tweet messages. 

\end{enumerate}

In the next laboratory assignment we will finish implementing a {\tt GatorTweet} program that can interact with Twitter. For
instance, we will add features that can authenticate a user with the Twitter servers, display a user's Twitter timeline, and post
a {\tt Tweet} to the Twitter site.

\section*{Summary of the Required Deliverables}

  This assignment invites you to submit one printed version of the following deliverables: 

  \begin{enumerate}
    \item A description of the meaning and purpose of the provided {\tt build.xml} file.
    \item A screenshot demonstrating that you can display and use the quickfix list in GVim.
    \item The final version of your {\tt Tweet.java} and {\tt GatorTweet.java} files.
    \item The output associated with running {\tt GatorTweet} with a sample ``Tweets.txt'' file.
  \end{enumerate}

  Along with turning in a printed version of these deliverables, you should ensure that everything is also available in the
  repository that is named according to the convention {\tt cs112S2014-<your user name>}. Please note that students in the class
  are responsible for completing and submitting their own version of this assignment.    While it is acceptable for members of
  this class to have high-level conversations, you should not share source code or full command lines with your classmates.
  Please see the instructor if you have questions about the policies for this laboratory assignment.

  \end{document}
