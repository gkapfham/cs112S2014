%!TEX root=cs112S2014-lab4.tex mainfile: cs112S2014-lab4.tex 

%!TEX root=cs112S2014-lab2.tex
% mainfile: cs112S2014-lab2.tex 
% Typical usage (all UPPERCASE items are optional):
%       \input 580pre
%       \begin{document}
%       \MYTITLE{Title of document, e.g., Lab 1\\Due ...}
%       \MYHEADERS{short title}{other running head, e.g., due date}
%       \PURPOSE{Description of purpose}
%       \SUMMARY{Very short overview of assignment}
%       \DETAILS{Detailed description}
%         \SUBHEAD{if needed} ...
%         \SUBHEAD{if needed} ...
%          ...
%       \HANDIN{What to hand in and how}
%       \begin{checklist}
%       \item ...
%       \end{checklist}
% There is no need to include a "\documentstyle."
% However, there should be an "\end{document}."
%
%===========================================================
\documentclass[11pt,twoside,titlepage]{article}
%%NEED TO ADD epsf!!
\usepackage{threeparttop}
\usepackage{graphicx}
\usepackage{latexsym}
\usepackage{color}
\usepackage{listings}
\usepackage{fancyvrb}
%\usepackage{pgf,pgfarrows,pgfnodes,pgfautomata,pgfheaps,pgfshade}
\usepackage{tikz}
\usepackage[normalem]{ulem}
\tikzset{
    %Define standard arrow tip
%    >=stealth',
    %Define style for boxes
    oval/.style={
           rectangle,
           rounded corners,
           draw=black, very thick,
           text width=6.5em,
           minimum height=2em,
           text centered},
    % Define arrow style
    arr/.style={
           ->,
           thick,
           shorten <=2pt,
           shorten >=2pt,}
}
\usepackage[noend]{algorithmic}
\usepackage[noend]{algorithm}
\newcommand{\bfor}{{\bf for\ }}
\newcommand{\bthen}{{\bf then\ }}
\newcommand{\bwhile}{{\bf while\ }}
\newcommand{\btrue}{{\bf true\ }}
\newcommand{\bfalse}{{\bf false\ }}
\newcommand{\bto}{{\bf to\ }}
\newcommand{\bdo}{{\bf do\ }}
\newcommand{\bif}{{\bf if\ }}
\newcommand{\belse}{{\bf else\ }}
\newcommand{\band}{{\bf and\ }}
\newcommand{\breturn}{{\bf return\ }}
\newcommand{\mod}{{\rm mod}}
\renewcommand{\algorithmiccomment}[1]{$\rhd$ #1}
\newenvironment{checklist}{\par\noindent\hspace{-.25in}{\bf Checklist:}\renewcommand{\labelitemi}{$\Box$}%
\begin{itemize}}{\end{itemize}}
\pagestyle{threepartheadings}
\usepackage{url}
\usepackage{wrapfig}
% removing the standard hyperref to avoid the horrible boxes
%\usepackage{hyperref}
\usepackage[hidelinks]{hyperref}
% added in the dtklogos for the bibtex formatting
\usepackage{dtklogos}
%=========================
% One-inch margins everywhere
%=========================
\setlength{\topmargin}{0in}
\setlength{\textheight}{8.5in}
\setlength{\oddsidemargin}{0in}
\setlength{\evensidemargin}{0in}
\setlength{\textwidth}{6.5in}
%===============================
%===============================
% Macro for document title:
%===============================
\newcommand{\MYTITLE}[1]%
   {\begin{center}
     \begin{center}
     \bf
     CMPSC 112\\Introduction to Computer Science II\\
     Spring 2014
     \medskip
     \end{center}
     \bf
     #1
     \end{center}
}
%================================
% Macro for headings:
%================================
\newcommand{\MYHEADERS}[2]%
   {\lhead{#1}
    \rhead{#2}
    %\immediate\write16{}
    %\immediate\write16{DATE OF HANDOUT?}
    %\read16 to \dateofhandout
    \def \dateofhandout {January 28, 2014}
    \lfoot{\sc Handed out on \dateofhandout}
    %\immediate\write16{}
    %\immediate\write16{HANDOUT NUMBER?}
    %\read16 to\handoutnum
    \def \handoutnum {3}
    \rfoot{Handout \handoutnum}
   }

%================================
% Macro for bold italic:
%================================
\newcommand{\bit}[1]{{\textit{\textbf{#1}}}}

%=========================
% Non-zero paragraph skips.
%=========================
\setlength{\parskip}{1ex}

%=========================
% Create various environments:
%=========================
\newcommand{\PURPOSE}{\par\noindent\hspace{-.25in}{\bf Purpose:\ }}
\newcommand{\SUMMARY}{\par\noindent\hspace{-.25in}{\bf Summary:\ }}
\newcommand{\DETAILS}{\par\noindent\hspace{-.25in}{\bf Details:\ }}
\newcommand{\HANDIN}{\par\noindent\hspace{-.25in}{\bf Hand in:\ }}
\newcommand{\SUBHEAD}[1]{\bigskip\par\noindent\hspace{-.1in}{\sc #1}\\}
%\newenvironment{CHECKLIST}{\begin{itemize}}{\end{itemize}}


\usepackage[compact]{titlesec}

\begin{document}
\MYTITLE{Laboratory Assignment Four: Implementing and Evaluating a Twitter Client}
\MYHEADERS{Laboratory Assignment Four}{Due: February 18, 2014}

\section*{Introduction}

As we continue to both refamiliarize ourselves with the fundamental characteristics of the object-oriented programming paradigm
and to practice using the some of the features of the Vim integrated development environment (IDE), we will take the final steps
towards implementing a complete Twitter client.  Along with learning how to use, enhance, and extend a Java class, we will also
review the steps that you must take to perform console input and output. This laboratory assignment represents the second of a
two-part series of assignments designed to review Java programming.

\section*{Accessing and Understanding GatorTweet}

To start this laboratory assignment, you should return to the {\tt cs112S2014-share} Git repository and type the command {\tt git
  pull} in the terminal window.  Now, you should have a {\tt lab4/GatorTweet/} directory that you can explore further.  Where is
the Java source code in this new directory? Why is it stored in the directory where you found it? Now, you should use GVim to
study the source code in the {\tt build.xml} file.  As you did for the past assignment, you can type the command {\tt :make} in
your GVim window and it will compile the three Java classes and save the bytecode in the correct subdirectories inside of the
{\tt bin/} directory.  Please see the instructor if you cannot get this to work.

\section*{Connecting to the Twitter Service}

\begin{sloppypar} At this point, you should look in the {\tt lib/} directory and notice that there is a file called {\tt
    twitter4j-core-3.0.5.jar}.  This file contains the additional Java classes that we will need to communicate with the Twitter
  service. Before you run any program that communicates with Twitter, you must use the {\tt export CLASSPATH} command to add this
  file to your {\tt CLASSPATH} variable.  To learn more about the library that we are using to connect to Twitter, please visit
  the Web site for the Twitter4J, available at \url{http://twitter4j.org/en/index.html}. Before you try to connect to Twitter,
  you should also study the {\tt twitter4j.properties} file. What does it contain? Why?  \end{sloppypar}

In advance of completing the next step, you should make sure that you have already created your Twitter account, followed some
other Twitter users, and tweeted through the Web-based interface. Now, you can go ahead and run the {\tt CreateProperties}
program in your terminal window.  You should follow this program's instructions to authorize your own Twitter client to access
your Twitter account.  As you complete this step, please make sure that you record the content of the {\tt twitter4j.properties}
file before and after running the {\tt CreateProperties} program.  Also, you should take a screen shot of the Web site that the
program asks you to visit. Finally, you can try to run the {\tt GetHomeTimeline} program.  What output did it produce? Were you
able to connect to the Twitter service? Students who could not complete this step should see the instructor.

\section*{Finishing the GatorTweet System}
  
First, you should add the code in the {\tt GetHomeTimeline} to your own {\tt GatorTweet} program, thus giving your system the
ability to display your own home timeline.  However, we still need to add some extra features to {\tt GatorTweet}!  For instance,
you should implement a feature that can post all of the valid tweets in the ``Tweets.txt'' file to your timeline.  If the {\tt
  Twitter} class in Twitter4J provides an instance method called {\tt updateStatus} what code do you need to write to post all of
the valid tweets to the Twitter system?  Students who want to learn more about the {\tt updateStatus} method can check the
\url{http://twitter4j.org/javadoc/} Web site.  Again, remember that you will need to use an iteration construct to iterate through
all of the valid tweets.  In order to fully implement this feature, you will also need to add a {\tt getMessage} method to {\tt
  Tweet.java}. Once you have posted each of the valid tweets to Twitter, your program should also output some debugging
information indicating that the tweet was correctly posted.  

  Right now, our {\tt GatorTweet} prgoram has a very limited user interface.  As one way to overcome this limitation, you can 
  implement a command-line interface that will allow the user to only read the home timeline, only update the status through the
  ``Tweets.txt'' file, or to do both.  In order to implement this feature, you will need to add conditional logic to your program.
  Finally, you should develop at least one additional feature of your choice.  For instance, you could implement a random tweet
  generator that will create a random tweet and then post it to Twitter. Alternatively, you could implement a feature that allows
  you to perform direct messaging on Twitter. As you finish the {\tt GatorTweet} system, please make sure that you add comments to
  all of the Java classes, including those that you did not modify or extend. You must add a comment to the header of every
  class and method, the declaration of every variable, and every other important line of code. 

\section*{Summary of the Required Deliverables}

  This assignment invites you to submit one printed version of the following deliverables: 

  \begin{enumerate}
    \item The contents of {\tt twitter4j.properties} before and after running {\tt CreateProperties}.
    \item A screenshot of the Web site that the {\tt CreateProperties} program asks you to visit.
    \item The final version of your {\tt GatorTweet.java} that implements all of the required features.
    \item The output from running {\tt GetHomeTimeline} when debugging mode is and is not enabled.
    \item The output from running {\tt GatorTweet} with all of the command lines that it supports.
    \item A reflective commentary on the challenges that you faced when implementing {\tt GatorTweet}.
  \end{enumerate}

  Along with turning in a printed version of these deliverables, you should ensure that everything is also available in
  the repository that is named according to the convention {\tt cs112S2014-<your user name>}. Please note that students
  in the class are responsible for completing and submitting their own version of this assignment.    While it is
  acceptable for members of this class to have high-level conversations, you should not share source code or full
  command lines with your classmates.  Deliverables that are nearly identical to the work of others will be taken as
  evidence of violating the \mbox{Honor Code}.  Please see the instructor if you have questions about the policies for
  this assignment.

  \end{document}
