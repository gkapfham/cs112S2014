%!TEX root=cs112S2014-lab4.tex
% mainfile: cs112S2014-lab4.tex 

\input{labspre.tex}

\usepackage[compact]{titlesec}

\begin{document}
\MYTITLE{Laboratory Assignment Four: Implementing and Evaluating a Twitter Client}
\MYHEADERS{Laboratory Assignment Four}{Due: February 18, 2014}

\section*{Introduction}

As we continue to both refamiliarize ourselves with the fundamental characteristics of the object-oriented programming paradigm
and to practice using the some of the features of the Vim integrated development environment (IDE), we will take the final steps
towards implementing a complete Twitter client.  Along with learning how to use, enhance, and extend a Java class, we will also
review the steps that you must take to perform console input and output. This laboratory assignment represents the second of a
two-part series of assignments designed to review Java programming.

\section*{Accessing and Understanding GatorTweet}

To start this laboratory assignment, you should return to the {\tt cs112S2014-share} Git repository and type the command {\tt git
  pull} in the terminal window.  Now, you should have a {\tt lab4/GatorTweet/} directory that you can explore further.  Where is
the Java source code in this new directory? Why is it stored in the directory where you found it? Now, you should use GVim to
study the source code in the {\tt build.xml} file.  As you did for the past assignment, you can type the command {\tt :make} in
your GVim window and it will compile the three Java classes and save the bytecode in the correct subdirectories inside of the
{\tt bin/} directory.  Please see the instructor if you cannot get this to work.

\section*{Connecting to the Twitter Service}

\begin{sloppypar}
  At this point, you should look in the {\tt lib/} directory and notice that there is a file called {\tt twitter4j-core-3.0.5.jar}.
  This file contains the additional Java classes that we will need to communicate with the Twitter service. Before you run any
  program that communicates with Twitter, you must use the {\tt export CLASSPATH} command to add this file to your {\tt CLASSPATH}
  variable.  To learn more about the library that we are using to connect to Twitter, please visit the Web site for a
  system called Twitter4J:
  \url{http://twitter4j.org/en/index.html}. Before you try to connect to Twitter, you should also study the {\tt
    twitter4j.properties} file. What does it contain? Why?
\end{sloppypar}

Before you complete the next step, you should make sure that you have already created your Twitter account, followed some other
Twitter users, and tweeted through the Web-based interface. Now, you should go ahead and run the {\tt CreateProperties} program in
your terminal window.  You should follow this program's instructions to authorize your own Twitter client to access your Twitter
account.  As you complete this step, please make sure that you record the content of the {\tt twitter4j.properties} file before
and after running the {\tt CreateProperties} program.  Also, you should take a screen shot of the Web site that the program asks
you to visit. Finally, you can try to run the {\tt GetHomeTimeline} program.  What output did it produce? Were you able to connect
to the Twitter service? Students who could not complete this step should see the instructor.

\section*{Finishing the GatorTweet System}

Now, you should implement a complete {\tt GatorTweet} class that performs the following operations in {\tt main}:


\section*{Summary of the Required Deliverables}

  This assignment invites you to submit one printed version of the following deliverables: 

  \begin{enumerate}
    \item A description of the meaning and purpose of the provided {\tt build.xml} file.
    \item A screenshot demonstrating that you can display and use the quickfix list in GVim.
    \item The final version of your {\tt Tweet.java} and {\tt GatorTweet.java} files.
    \item The output associated with running {\tt GatorTweet} with a sample ``Tweets.txt'' file.
  \end{enumerate}

  Along with turning in a printed version of these deliverables, you should ensure that everything is also available in the
  repository that is named according to the convention {\tt cs112S2014-<your user name>}. Please note that students in the class
  are responsible for completing and submitting their own version of this assignment.    While it is acceptable for members of
  this class to have high-level conversations, you should not share source code or full command lines with your classmates.
  Please see the instructor if you have questions about the policies for this laboratory assignment.

  \end{document}
