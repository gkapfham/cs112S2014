\href{http://www.cs.allegheny.edu/sites/cs112S2013/125}{}

\subsection{Lab 6: Using the Stack Abstract Data Type to Implement an
Arithmetic Interpreter}

\subsection{\emph{Assigned: Tuesday, March 26, 2013}}

\subsection{\emph{Due: Wednesday, April 3, 2013}}

\begin{center}\rule{3in}{0.4pt}\end{center}

\subsection{Purpose}

To familiarize ourselves with the use of the Stack abstract data type.
Specifically, to implement a simple \texttt{StackMachine} that will
support a primitive language that can perform arithmetic with a Stack.
Optionally, to explore the implementation of a test suite for the Stack
abstract data type.

\subsection{Deliverables}

In order to satisfactorily complete this laboratory assignment, you must
submit the following deliverables. The laboratory notebook deliverable
must be posted on a course Web site page for this laboratory. There must
be a link to your laboratory notebook at the page that is linked to at
the course web site page for this laboratory. Students are encouraged to
record additional insights about the laboratory session in this notebook
area. Failure to include a laboratory notebook will result in the
deduction of points. Please see the Instructor if you do not understand
these instructions. Each of the printed code segments must include a
signed pledge on the document. When you submit the laboratory
assignment, please use your laboratory notebook as the cover sheet for
your source code and output.

\paragraph{Laboratory Notebook}

\begin{enumerate}
\item
  Please comment on some problem that can be solved by using the Stack
  abstract data type. Your comments should include a brief introduction
  to the problem and a discussion of how the Stack abstract data type is
  incorporated in your solution. You can document a problem that you
  have developed where the use of a Stack would be beneficial.
  Alternatively, you can search the Internet to find some well-known
  problems that require the use of a Stack. You might consider
  discussing the ``infix to postfix'' conversion example. After
  searching the Internet for details about this example, please see the
  Instructor if you are interested in learning more about it.
\item
  A review of each pitfall or problem that you encountered during the
  laboratory and a discussion of the techniques that you used to
  overcome this difficulty. If no difficulties were encountered, then
  please include a record of important commands, notes, etc. that you
  think will be useful during later laboratories.
\end{enumerate}

\paragraph{Source Code and Output}

\begin{enumerate}
\item
  The source code listing for the class \texttt{StackMachine} that is
  included in the file \texttt{StackMachine.java}. Your implementation
  must meet the specifications outlined below.
\item
  The output from the execution of the \texttt{StackMachine} class. Your
  output should clearly demonstrate the functionality of each operator
  that the \texttt{StackMachine} supports. You do not need to
  demonstrate all unique combinations of \texttt{StackMachine}
  functionality.

  Your Java classes must fully meet the specifications that are outlined
  later in the laboratory assignment. Furthermore, they must adhere to
  the following stylistic conventions.
\end{enumerate}

\begin{enumerate}
\item
  There must be comments at the beginning with your name and the word
  ``PLEDGE'' (you must sign the program here before you hand in the
  printout). Following this, there should be comments briefly describing
  what the program does. Use the standard JavaDoc commenting tags (like
  \texttt{@author} inside of a \texttt{/** ... */} comment block).
\item
  There should be comments as needed throughout the program to explain
  variables and summarize the important points in your program. For
  example, when you declare a variable, you should include a comment
  that describes the purpose of this variable. When you assign a new
  value to a variable, you should include a comment to describe the
  intended purpose of the assignment.
\item
  The indenting scheme of your Java program must be consistent-all
  ``\{\ldots{}\}'' pairs should be aligned, all statements inside
  ``\{\ldots{}\}'' should be indented the same amount. Remember,
  \texttt{emacs} can help with this!
\item
  Use blank lines in your program to improve readability and to separate
  things into logical groupings of statements. For instance, it makes
  sense to group your variable declarations together and separate them
  from other statements.
\item
  Use Courier or another fixed-width font for the program and the
  output.
\item
  All of your Java code and output must be printed so that it has the
  date \emph{on which the code was printed} at the top of the page. If
  you are having trouble printing, then please see the Instructor.
\end{enumerate}

\subsection{Software Development Assignment}

A machine with only a single stack for storage is called a
\texttt{StackMachine}. A stack machine can have a primitive language
that describes the operations that it can perform. Consider the
following primitive language for a stack machine:

\ctable[pos = H, center, botcap]{ll}
{% notes
}
{% rows
\FL
Command & Meaning
\\\noalign{\medskip}
\emph{int} & push \emph{int} on the stack
\\\noalign{\medskip}
+ & push a `+' on the stack
\\\noalign{\medskip}
s & push an 's' on the stack
\\\noalign{\medskip}
e & evaluate the top of the stack (see below)
\\\noalign{\medskip}
d & display contents of the stack
\\\noalign{\medskip}
x & stop the stack machine
\LL
}

The `d' command simply prints out the contents of the stack, one element
per line, beginning with the top of the stack. The behavior of the `e'
command depends on the contents of the stack when the `e' is issued. The
following describes the rules that the `e' command must follow:

\begin{enumerate}
\item
  If `+' is on the top of the stack, then the `+' is popped off the
  stack, the following two integers are popped and added, and the result
  is pushed back on the stack.
\item
  If 's' is on top of the stack, then the 's' is popped off the stack
  and the following two items are swapped on the stack.
\item
  If an integer is on the top of the stack, or the stack is empty, then
  the stack is left unchanged.
\end{enumerate}

The following examples illustrates the impact of the `e' command in
various situations. Note that in these examples, the top of the stack is
on the left.

Stack Before

Stack After

\begin{verbatim}
+ 1 2 5 s ...
\end{verbatim}

\begin{verbatim}
3 5 s ...
\end{verbatim}

\begin{verbatim}
s 1 + + 99 ...
\end{verbatim}

\begin{verbatim}
+ 1 + 99 ...
\end{verbatim}

\begin{verbatim}
1 + 3 ...
\end{verbatim}

\begin{verbatim}
1 + 3 ...
\end{verbatim}

You must implement an interpreter for this simple language. Input to the
program is a series of commands, with one per line. The interpreter
should prompt for commands with
\texttt{\textgreater{}\textgreater{}\textgreater{}\textgreater{}\textgreater{}}
or something else appropriate. You do not need to do any error checking.
That is, you can assume that all commands are valid and that the
appropriate number of arguments are on the stack for evaluation. You can
implement a sensible exception handling system for extra credit.
Finally, be aware of the fact that you will have to decide what types of
variables are going to be pushed on and popped off the stack. This
choice may impact the types of casting that you will perform inside of
your \texttt{StackMachine}.

You should implement this system in an object-oriented fashion and your
\texttt{StackMachine} may use an instance of \texttt{java.util.Stack}.
Yet, you are encouraged to implement your own \texttt{Stack}, using the
examples provided in the textbook. However, implementation of your own
stack is not required for the purposes of this laboratory. You should
consider the use of \texttt{ConsoleReader} or \texttt{Scanner} in the
implementation of your \texttt{StackMachine}. Make sure that the header
of each class and every method of each class is commented in an
appropriate fashion. You must use JavaDoc comments to describe the
methods inside of your classes.

Students will be awarded extra credit if they implement a test suite for
the one or more of the data types \texttt{java.util.Stack},
\texttt{ArrayStack}, or \texttt{NodeStack}. If you implement a test
suite for \texttt{java.util.Stack}, it should be called
\texttt{TestStack} and it should adhere to the JUnit testing framework
(similar naming conventions would apply to the test suites for the other
Stack implementations). A test suite should contain at least ten
individual test cases that test some functionality of the Stack abstract
data type (ADT). One example of a test case for any Stack implementation
would involve pushing multiple data items onto the Stack and then
verifying that the top of the stack is correct. If you were testing the
\texttt{ArrayStack}, you should also consider tests that check to ensure
that it throws the appropriate exceptions at the appropriate time (e.g.,
when the Stack is full and a \texttt{push} is performed or when the
Stack is empty and a \texttt{pop} is executed). Please see the
Instructor if you would like to learn more about writing test cases with
JUnit. Any student who finds a substantial defect in the
\texttt{java.util.Stack} from the Java library will be awarded
additional extra credit.

\begin{center}\rule{3in}{0.4pt}\end{center}

\href{http://creativecommons.org/licenses/by-nc/1.0/}{\includegraphics{http://creativecommons.org/images/public/somerights.gif}}
Unless otherwise specified, this work is licensed under a
\href{http://creativecommons.org/licenses/by-nc/1.0/}{Creative Commons
License},

\begin{center}\rule{3in}{0.4pt}\end{center}
