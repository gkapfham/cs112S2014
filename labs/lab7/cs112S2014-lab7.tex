%!TEX root=cs112S2014-lab7.tex 
% mainfile: cs112S2014-lab7.tex 

\input{labspre.tex}

\usepackage[compact]{titlesec}

\begin{document} \MYTITLE{Laboratory Assignment Seven: Performing Arithmetic Interpretation with a Stack}
\MYHEADERS{Laboratory Assignment Seven}{Due: April 2, 2014}

\section*{Introduction}

  Many real-world programs use the stack abstract data type (ADT) to solve a problem.  For instance, the Java virtual
  machine (JVM) relies on a stack to store the activation records for the methods that are called during the execution
  of a program.  Stacks are often implemented as part of the hardware used in many modern computers. In this laboratory
  assignment, you will learn how to use the stack ADT in a Java program.  In particular, you will implement and test a
  {\tt StackMachine} that provides a simple language supporting arithmetic operations.

\section*{Learning About the Stack}

  Before you start to implement your {\tt StackMachine}, you should take some time to learn more about the Stack ADT.
  What is an example of a progrem that can be solved using the stack? Your response to this question should include a
  brief introduction to the problem and a discussion of how the stack is incorporated into the solution. For instance,
  you could explain a program that you have already implemented that used a stack to solve a key problem. Alternatively,
  you can study books, articles, and online sources to investigate a well-known problem that can be solved through the
  use of a stack.  For instance, you could study the algorithm that converts an expression in infix notation to postfix
  notation. Please see the instructor if you have questions about this part of the assignment.

\section*{Implementing an Arithmetic Interpreter}
  
  In this assignment, you will write a Java program that constructs an instance of the {\tt java.util.Stack} class. What
  are the key methods are provided by {\tt java.util.Stack}? What are the inputs and outputs of the key methods? What is
  the behavior of these methods? Please make sure that you review your text book and consult the instructor if you do
  not understand the methods of \mbox{the stack}.
  
  A program that only uses a single stack for storage sometimes is called a stack machine. A stack machine can support a
  language describing the operations that it can perform. Consider the following language for a stack machine that
  supports select arithmetic computations:

\begin{tabular}{r | l}
Command & Meaning \\ \hline
\emph{int} & push \emph{int} on the stack \\
+ & push a ``+'' on the stack \\
s & push an ``s'' on the stack \\
e & evaluate the top of the stack (see below) \\
d & display contents of the stack \\
x & stop the stack machine \\ 
\end{tabular}

  The ``d'' command simply prints out the contents of the stack, one element per line, beginning with the top of the
  stack. The behavior of the ``e'' command depends on the contents of the stack when the ``e'' is issued. For this
  laboratory assignment, the ``e'' command must follow these rules:

\begin{enumerate}

  \item If ``+'' is on the top of the stack, then the ``+'' is popped off the stack, the following two integers are popped
    and added, and the result is pushed back on the stack.  
  
  \item If ``s'' is on top of the stack, then the ``s'' is popped off the stack and the following two items are swapped
    on the stack.  
  
  \item If an integer is on the top of the stack, or the stack is empty, then the stack is left unchanged.  

\end{enumerate}

The following examples illustrates the impact of the ``e'' command in various situations. Note that in these examples,
the top of the stack is on the left.


\section*{Summary of the Required Deliverables}

  This assignment invites you to submit a signed and printed version of the following deliverables: 

  \begin{enumerate} 
  \itemsep0pt
  \item A description of all of the command-line arguments supported by your benchmarking tool 

  \item A sample output from running {\tt SortingExperiment} in all of its relevant configurations

  \item The final version of the commented source code for the entire benchmarking framework 

  \item A comprehensive written report that fully explains the results of your experimental study

  \item A reflective commentary on the challenges that you faced when conducting the experiments
   
  \end{enumerate}

  Along with turning in a printed version of these deliverables, you should ensure that everything is also available in
  the repository that is named according to the convention {\tt cs112S2014-<your user name>}. Please note that students
  in the class are responsible for completing and submitting their own version of this assignment.    While it is
  acceptable for members of this class to have high-level conversations, you should not share source code or full
  command lines with your classmates.  Deliverables that are nearly identical to the work of others will be taken as
  evidence of violating the \mbox{Honor Code}.  Please see the instructor if you have questions about the policies for
  this assignment.

  \end{document}
