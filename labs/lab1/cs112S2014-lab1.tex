%!TEX root=cs112S2014-lab1.tex
% mainfile: cs112S2014-lab1.tex 

%!TEX root=cs112S2014-lab2.tex
% mainfile: cs112S2014-lab2.tex 
% Typical usage (all UPPERCASE items are optional):
%       \input 580pre
%       \begin{document}
%       \MYTITLE{Title of document, e.g., Lab 1\\Due ...}
%       \MYHEADERS{short title}{other running head, e.g., due date}
%       \PURPOSE{Description of purpose}
%       \SUMMARY{Very short overview of assignment}
%       \DETAILS{Detailed description}
%         \SUBHEAD{if needed} ...
%         \SUBHEAD{if needed} ...
%          ...
%       \HANDIN{What to hand in and how}
%       \begin{checklist}
%       \item ...
%       \end{checklist}
% There is no need to include a "\documentstyle."
% However, there should be an "\end{document}."
%
%===========================================================
\documentclass[11pt,twoside,titlepage]{article}
%%NEED TO ADD epsf!!
\usepackage{threeparttop}
\usepackage{graphicx}
\usepackage{latexsym}
\usepackage{color}
\usepackage{listings}
\usepackage{fancyvrb}
%\usepackage{pgf,pgfarrows,pgfnodes,pgfautomata,pgfheaps,pgfshade}
\usepackage{tikz}
\usepackage[normalem]{ulem}
\tikzset{
    %Define standard arrow tip
%    >=stealth',
    %Define style for boxes
    oval/.style={
           rectangle,
           rounded corners,
           draw=black, very thick,
           text width=6.5em,
           minimum height=2em,
           text centered},
    % Define arrow style
    arr/.style={
           ->,
           thick,
           shorten <=2pt,
           shorten >=2pt,}
}
\usepackage[noend]{algorithmic}
\usepackage[noend]{algorithm}
\newcommand{\bfor}{{\bf for\ }}
\newcommand{\bthen}{{\bf then\ }}
\newcommand{\bwhile}{{\bf while\ }}
\newcommand{\btrue}{{\bf true\ }}
\newcommand{\bfalse}{{\bf false\ }}
\newcommand{\bto}{{\bf to\ }}
\newcommand{\bdo}{{\bf do\ }}
\newcommand{\bif}{{\bf if\ }}
\newcommand{\belse}{{\bf else\ }}
\newcommand{\band}{{\bf and\ }}
\newcommand{\breturn}{{\bf return\ }}
\newcommand{\mod}{{\rm mod}}
\renewcommand{\algorithmiccomment}[1]{$\rhd$ #1}
\newenvironment{checklist}{\par\noindent\hspace{-.25in}{\bf Checklist:}\renewcommand{\labelitemi}{$\Box$}%
\begin{itemize}}{\end{itemize}}
\pagestyle{threepartheadings}
\usepackage{url}
\usepackage{wrapfig}
% removing the standard hyperref to avoid the horrible boxes
%\usepackage{hyperref}
\usepackage[hidelinks]{hyperref}
% added in the dtklogos for the bibtex formatting
\usepackage{dtklogos}
%=========================
% One-inch margins everywhere
%=========================
\setlength{\topmargin}{0in}
\setlength{\textheight}{8.5in}
\setlength{\oddsidemargin}{0in}
\setlength{\evensidemargin}{0in}
\setlength{\textwidth}{6.5in}
%===============================
%===============================
% Macro for document title:
%===============================
\newcommand{\MYTITLE}[1]%
   {\begin{center}
     \begin{center}
     \bf
     CMPSC 112\\Introduction to Computer Science II\\
     Spring 2014
     \medskip
     \end{center}
     \bf
     #1
     \end{center}
}
%================================
% Macro for headings:
%================================
\newcommand{\MYHEADERS}[2]%
   {\lhead{#1}
    \rhead{#2}
    %\immediate\write16{}
    %\immediate\write16{DATE OF HANDOUT?}
    %\read16 to \dateofhandout
    \def \dateofhandout {January 28, 2014}
    \lfoot{\sc Handed out on \dateofhandout}
    %\immediate\write16{}
    %\immediate\write16{HANDOUT NUMBER?}
    %\read16 to\handoutnum
    \def \handoutnum {3}
    \rfoot{Handout \handoutnum}
   }

%================================
% Macro for bold italic:
%================================
\newcommand{\bit}[1]{{\textit{\textbf{#1}}}}

%=========================
% Non-zero paragraph skips.
%=========================
\setlength{\parskip}{1ex}

%=========================
% Create various environments:
%=========================
\newcommand{\PURPOSE}{\par\noindent\hspace{-.25in}{\bf Purpose:\ }}
\newcommand{\SUMMARY}{\par\noindent\hspace{-.25in}{\bf Summary:\ }}
\newcommand{\DETAILS}{\par\noindent\hspace{-.25in}{\bf Details:\ }}
\newcommand{\HANDIN}{\par\noindent\hspace{-.25in}{\bf Hand in:\ }}
\newcommand{\SUBHEAD}[1]{\bigskip\par\noindent\hspace{-.1in}{\sc #1}\\}
%\newenvironment{CHECKLIST}{\begin{itemize}}{\end{itemize}}


\usepackage[compact]{titlesec}

\begin{document}
\MYTITLE{Laboratory Assignment One: Version Control with Git and Bitbucket}
\MYHEADERS{Laboratory Assignment One}{Due: January 28, 2014}

\section*{Introduction}

Practicing software developers normally use a version control system to manage most of the artifacts produced during the
phases of the software development life cycle.  In this course, we will always use the Git distributed version control
system to manage the files associated with our laboratory assignments.  In this laboratory assignment, you will learn
how to use the Bitbucket service for managing Git repositories and the {\tt git} command-line tool in the Ubuntu Linux
operating system. After connecting to the course's Git repository and creating your own repository, you will compile and
run two Java programs, write about their output, and commit your final report to a repository.

\section*{Configuring Git and Bitbucket}

During this laboratory assignment and subsequent assignments, we will securely communicate with the Bitbucket.org
servers that will host our all of our projects.  In this laboratory assignment, we will perform all of the steps to
configure the accounts on the departmental servers and the Bitbucket service.  Throughout the assignment, you should
refer to the following Web site for additional information:
\url{https://confluence.atlassian.com/display/BITBUCKET/Bitbucket+101}.  As you will be required to turn in a report 
describing each step that you finish in this assignment, please be sure to keep a record of all of the steps that you
complete and the challenges that you face.  You are also responsible for working with a team to ensure that
each member of the team is able to successfully complete each of the steps outlined in this assignment.

\begin{enumerate}
	
  \item If you have never done so before, you must use the {\tt ssh-keygen} program to create secure-shell keys that you
    can use to support your communication with the Bitbucket servers.  Type {\tt man ssh-keygen} and talk with the
    members of your team to learn more about how to use this program.  What files does {\tt ssh-keygen} produce?  Where
    does this program store these files?

  \item If you do not already have a Bitbucket account, please go to the Bitbucket Web site and create one --- 
    make sure that you use your {\tt allegheny.edu} email address so that you can create an unlimited number of free
    Bitbucket repositories. Then, upload your ssh key to Bitbucket.

  \item Now, you need to test to see if you can authenticate with the Bitbucket servers.  First, show the course
    instructor that you have correctly configured your Bitbucket account.  Now, ask the instructor to share the course's
    Git repository with you.  Open a terminal window on your workstation and change into the directory where you will
    store your files for this course.  For instance, you might make a {\tt cs112S2014/} directory that will contain the
    Git repository that I will always use to share files with you.  Once you have done so, please type the following
    command: {\tt git clone git@bitbucket.org:gkapfham/cs112s2014-share.git}.  If everything worked correctly, you
    should be able to download all of the files that you will need to use for this laboratory assignment. Please resolve
    any problems that you encountered by first reviewing the Bitbucket documentation and then discussing the matter
    with your team. If you are still not able to run the {\tt git clone} command, then please see the instructor.

  \item Using your terminal window, you should browse the files that are in this Git repository.  In particular, please
    look in the {\tt labs/lab1/src/} directory and use Vim to study the two Java programs that you find.  Remember, the
    {\tt cd} command allows you to change into a directory. 

\end{enumerate}

\section*{Creating a New Repository}

Now that you have learned how to clone an existing Git repository, you should make a new repository in the {\tt
cs112S2014/} directory that you previously created.  First, make a new directory called {\tt cs112S2014-<your user
name>}. Then, change into this directory and type the command {\tt git init .}.  At this point, you should go into the
{\tt cs112s2014-share} repository and use the {\tt cp -r} command to copy the entire {\tt labs/} directory from the {\tt
cs112s2014-share} repository to {\tt cs112S2014-<your user name>}.  Once the files are in your own Git repository,
please use the {\tt git add} and {\tt git commit} commands to add them correctly. If you do not know how to use the {\tt
git add} and {\tt git commit} commands in the terminal window, please learn more about them by searching on the
Internet, talking about them with your team, and discussing them with the course instructor.

Next, you should use the Bitbucket Web site to create a repository that has the same name as the local directory and
local repository.  You must follow Bitbucket's instructions to push the code and tags in your local repository to the
remote one. When you are finished with this step, you should see in your Web browser that the Bitbucket servers are
storing the two Java programs. Once the Git repository contains the correct files, you should share your Bitbucket
repository with the course instructor, whose my Bitbucket user name is ``gkapfham''.

At this point, you can learn more about Git by consulting Web sites like \url{http://try.github.io/} and
\url{http://gitimmersion.com/}.  At minimum, you should ensure that you fully understand how to use the following Git
commands in the terminal window:

\begin{enumerate} 
			
	\item {\tt git init}

	\item {\tt git status}

	\item {\tt git add} 

	\item {\tt git commit}

	\item {\tt git push}

	\item {\tt git pull} 

\end{enumerate}

\section*{Compiling, Running, and Understanding Java Programs}

  Once you have mastered the use of Git and version control, you should return to the {\tt labs/lab1/src/} directory that
  contains the two Java programs. Now, use the Java compiler to compile the {\tt Hooray.java} program.  That is, you
  should type {\tt javac Hooray.java} in the terminal window.  Next, you can run this program by typing {\tt java
  Hooray} in the terminal window.  What output does this program produce?  Why does it create this output? How do you
  stop this program? 

  After compiling, using, and studying the {\tt Hooray} program, you should complete the same steps for the {\tt
  Weeee.java} program. Go ahead and compile and run this program.  What output does it produce? Why does it create this
  output? How is the output similar to and different from that which was created by the {\tt Hooray} program? Once you
  have finished studying and understanding these two programs, add comments to the code to explain what they do and how
  they work. Finally, please make sure that the commented version of each program is correctly committed to your Git
  version control repository hosted by Bitbucket.
  
  % After you
  % are done running both of these programs, you should write a short one-page report that explains:

\section*{Summary of the Required Deliverables}

  This assignment invites you to submit one printed version of the following deliverables:

  \vspace*{-.05in}
  \begin{enumerate}
	
    \item A description of the steps that a user must take to configure Git and Bitbucket

    \item A description of the inputs, outputs, and behavior of the six aforementioned Git commands

    \item A commented version of the {\tt Hooray.java} and {\tt Weeee.java} programs
	
    \item A one-page report that clearly responds to the following four prompts:
  
      \vspace*{-.05in}
      \begin{enumerate}

	\item The steps that you took to compile and run both of these programs

	\item The output that each of these programs produce

	\item An explanation for why these programs create the output that they do

	\item A discussion of the similarities and differences between these programs and their output

      \end{enumerate}

  \end{enumerate}

Before you turn in this assignment, you also must ensure that the course instructor has read access to your Bitbucket
repository that is named according to the convention {\tt cs112S2014-<your user name>}. Please note that each student in
the class is responsible for completing and submitting their own version of this assignment.  However, you also will be
assigned to work to a team that is tasked with ensuring that all of its members are able to complete each step of the
assignment.  Team members should make themselves available to each other to answer questions and resolve any problems
that develop during the laboratory session. While it is acceptable for members of a team to have high-level
conversations, you should not share source code or full command lines with your team members. To ensure that you can
communicate effectively, members of each team should sit next to each other in the room.  Please see the instructor if
you have questions about this policy.



\end{document}
