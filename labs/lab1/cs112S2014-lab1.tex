%!TEX root=cs112S2014-lab1.tex
% mainfile: cs112S2014-lab1.tex 

\input{labspre.tex}

\usepackage[compact]{titlesec}

\begin{document}
\MYTITLE{Laboratory Assignment One: Version Control with Git and Bitbucket}
\MYHEADERS{Laboratory Assignment One}{Due: January 28, 2014}

\section*{Introduction}

Practicing software developers normally use a version control system to manage most of the artifacts produced during the
phases of the software development life cycle.  In this course, we will always use the Git distributed version control
system to manage the files associated with our laboratory assignments.  In this laboratory assignment, you will learn
how to use the Bitbucket service for managing Git repositories and the {\tt git} command-line tool in the Ubuntu Linux
operating system. After connecting to the course's Git repository and creating your own repository, you will compile and
run two Java programs, write about their output, and commit your final report to a repository.

\section*{Configuring Git and Bitbucket}

During this laboratory assignment and subsequent assignments, we will securely communicate with the Bitbucket.org
servers that will host our all of our projects.  In this laboratory assignment, we will perform all of the steps to
configure the accounts on the departmental servers and the Bitbucket service.  Throughout the assignment, you should
refer to the following Web site for additional information:
\url{https://confluence.atlassian.com/display/BITBUCKET/Bitbucket+101}.  As you will be required to turn in a report 
describing each step that you finish in this assignment, please be sure to keep a record of all of the steps that you
complete and the challenges that you face.  You are also responsible for working with your partner to ensure that
this individual also is able to successfully complete each of the steps outlined in this assignment.

\begin{enumerate}
	
  \item If you have never done so before, you must use the {\tt ssh-keygen} program to create secure-shell keys that you
    can use to support your communication with the Bitbucket servers.  Type {\tt man ssh-keygen} and talk with the
    members of your team to learn more about how to use this program.  What files does {\tt ssh-keygen} produce?  Where
    does this program store these files?

  \item If you do not already have a Bitbucket account, please go to the Bitbucket Web site and create one --- 
    make sure that you use your {\tt allegheny.edu} email address so that you can create an unlimited number of free
    Bitbucket repositories.

  \item Now, you need to test to see if you can authenticate with the Bitbucket servers.  First, show the course
    instructor that you have correctly configured your Bitbucket account.  Now, ask instructor to share the
    course's Git repository with you.  Open a terminal window on your workstation and change into the directory where
    you will store your files for this course.  For instance, you might make a {\tt cs112S2014/} directory that will
    contain the Git repository that I will always use to share files with you. Open a terminal window on your
    workstation and create and then change into the directory where you will store your files for this laboratory
    assignment.  Then, please type the following command: {\tt git clone
    git@bitbucket.org:gkapfham/cs112s2014-share.git}.  If everything worked correctly, you should be able to download
    all of the files that you will need to use for this laboratory assignment. Please resolve any problems that your
    encountered by first reviewing the Bitbucket documentation and then discussing the matter with your partner.

  \item Using your terminal window, you should browse the files that are in this Git repository. Remember,
    the {\tt cd} command allows you to change into a directory.  In particular, please look in the {\tt labs/lab1/src/}
    directory and use Vim to study the two Java programs that you find.

\end{enumerate}

\section*{Creating a New Repository}

Now that you have learned how to clone an existing Git repository, you should create a new repository in the {\tt
cs112S2014/} directory that you previously created.  First, create a new directory called {\tt cs112S2014-<your user
name>}. Then, change into this directory and type the command {\tt git init .}.  At this point, you should go into the
{\tt cs580s2014-share} repository and use the {\tt cp -r} command to copy the entire {\tt labs/} directory from the {\tt
cs580s2014-share} repository to {\tt cs112S2014-<your user name>}.  Once the files are in your own Git repository,
please use the {\tt git add} and {\tt git commit} commands to add them correctly. If you do not know how to use the {\tt
git add} and {\tt git commit} commands in the terminal window, please learn more about them by searching on the
Internet, talking about them with your partner, and discussing them with the course instructor.

Next, you should use the Bitbucket Web site to create a repository that has the same name as the local directory and
local repository.  You must follow Bitbucket's instructions to push the code and tags in your local repository to the
remote one. When you are finished with this step, you should see in your Web browser that the Bitbucket servers are
storing the two Java programs. Once the Git repository contains the correct files, you should share your Bitbucket
repository with the course instructor (my Bitbucket user name is ``gkapfham'').

At this point, you can learn more about Git by consulting Web sites like \url{http://try.github.io/} and
\url{http://gitimmersion.com/}.  At minimum, you should ensure that you fully understand how to use the following Git
commands in the terminal window:

\begin{enumerate} 
			
	\item {\tt git init}

	\item {\tt git status}

	\item {\tt git add} 

	\item {\tt git commit}

	\item {\tt git push}

	\item {\tt git pull} 

\end{enumerate}

\section*{Compiling, Running, and Understanding Java Programs}

\section*{Summary of the Required Deliverables}

This assignment invites your team to submit one printed version of a tutorial that contains:

\begin{enumerate}
	
	\item A description of the steps that a user must take to configure Git and Bitbucket

	\item A description of the inputs, outputs, and behavior of the aforementioned Git commands

\end{enumerate}

You must also ensure that the instructor has read access to your Bitbucket repository that is named according to the
convention {\tt cs290F2013-lab1-team{\em k}}, with {\tt {\em k}} representing the number of your assigned team. Please
see the instructor if you would like to print your tutorial slides in color.

\end{document}
