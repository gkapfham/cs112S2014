%!TEX root=cs112S2014-lab6.tex 
% mainfile: cs112S2014-lab6.tex 

\input{labspre.tex}

\usepackage[compact]{titlesec}

\begin{document} \MYTITLE{Laboratory Assignment Six: Evaluating the Performance of Sorting Algorithms}
\MYHEADERS{Laboratory Assignment Six}{Due: March 25, 2014}

\section*{Introduction}

Sorting algorithms are commonly used in a wide variety of practical computer applications.  In this laboratory
assignment, we will learn more about sorting while also continuing to familiarize ourselves with the empirical analysis
of algorithms.  In particular, we will analyze the performance of five sorting algorithms as they are applied to
randomly generated input data sets.  Students who would like to learn more about sorting algorithms are encouraged to
visit the following Web site: \url{https://www.cs.usfca.edu/~galles/visualization/ComparisonSort.html}.

\section*{Accessing and Using the Sorting Benchmarking Framework}

To start this laboratory assignment, you should return to the {\tt cs112S2014-share} Git repository and type the {\tt
git pull} command in the terminal window.  Now, you should have a {\tt lab6/} directory that you can explore further.
Once again, please make sure that you can find the source code in this new directory and you understand why the
directories in the assignment are structured the way that they are. Now, you should use GVim to study the source code in
the {\tt build.xml} file.  As in the past assignments, when editing a Java program you can type the command {\tt :make}
in your GVim window and it will compile the all of the Java classes and save the bytecode in the correct subdirectories
inside of the {\tt bin/} directory.  Please see the instructor if you cannot get this to work.


\section*{Incorporating Additional Features}

\section*{Conducting Experiments to Evaluate Efficiency}

\section*{Summary of the Required Deliverables}

  This assignment invites you to submit a signed and printed version of the following deliverables: 

  \begin{enumerate} 
  \itemsep0pt
  \item Using diagram(s), an explanation of how recursion works in the Java programming language

  \item A short discussion of the different primitive types that are available in the Java language

  \item The final version of the commented source code for your Fibonacci benchmarking framework

  \item A comprehensive written report that fully explains the results of your experimental study

  \item A reflective commentary on the challenges that you faced when conducting the experiments
   
  \end{enumerate}

  Along with turning in a printed version of these deliverables, you should ensure that everything is also available in
  the repository that is named according to the convention {\tt cs112S2014-<your user name>}. Please note that students
  in the class are responsible for completing and submitting their own version of this assignment.    While it is
  acceptable for members of this class to have high-level conversations, you should not share source code or full
  command lines with your classmates.  Deliverables that are nearly identical to the work of others will be taken as
  evidence of violating the \mbox{Honor Code}.  Please see the instructor if you have questions about the policies for
  this assignment.

  \end{document}
