%!TEX root=cs112S2014-lab5.tex 
% mainfile: cs112S2014-lab5.tex 

\input{labspre.tex}

\usepackage[compact]{titlesec}

\begin{document} \MYTITLE{Laboratory Assignment Five: Evaluating the Performance of Iteration and Recursion}
\MYHEADERS{Laboratory Assignment Five}{Due: February 25, 2014}

\section*{Introduction}

The Fibonacci sequence are the numbers in the in the integer sequence that develops in the following fashion:
$0, 1, 1, 2, 3, 5, 8, 13, 21, 34, 55, 89, \ldots$. More formally, it is known that the {\em n}-th Fibonacci number,
denoted $F_n$, is defined by the following equation $F_n = F_{n-1} + F_{n-2}$.

\section*{Accessing the Fibonacci Benchmark}

\section*{Conducting Experiments to Evaluate Efficiency}

\section*{Summary of the Required Deliverables}

  This assignment invites you to submit one printed version of the following deliverables: 

  \begin{enumerate} 
    
    \item A reflective commentary on the challenges that you faced when conducting the experiments
   
  \end{enumerate}

  Along with turning in a printed version of these deliverables, you should ensure that everything is also available in
  the repository that is named according to the convention {\tt cs112S2014-<your user name>}. Please note that students
  in the class are responsible for completing and submitting their own version of this assignment.    While it is
  acceptable for members of this class to have high-level conversations, you should not share source code or full
  command lines with your classmates.  Deliverables that are nearly identical to the work of others will be taken as
  evidence of violating the \mbox{Honor Code}.  Please see the instructor if you have questions about the policies for
  this assignment.

  \end{document}
