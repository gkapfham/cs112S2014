%!TEX root=cs112S2014-lab5.tex 
% mainfile: cs112S2014-lab5.tex 

\input{labspre.tex}

\usepackage[compact]{titlesec}

\begin{document} \MYTITLE{Laboratory Assignment Five: Evaluating the Performance of Iteration and Recursion}
\MYHEADERS{Laboratory Assignment Five}{Due: February 25, 2014}

\section*{Introduction}

The Fibonacci sequence are the numbers in the in the integer sequence that develops in the following fashion: $0, 1, 1,
2, 3, 5, 8, 13, 21, 34, 55, 89, \ldots$. More formally, it is known that the {\em n}-th Fibonacci number, denoted $F_n$,
is defined by the following equation $F_n = F_{n-1} + F_{n-2}$. In this laboratory assignment, we will explore and
extend an implementation of iterative and recursive algorithms for calculating the numbers in the Fibonacci sequence.
Moreover, we will investigate how the bit-depth of the variable that stores the output of the Fibonacci calculator can
influence the correctness of the final result. After learning how to conduct a detailed experimental study of an
algorithm, students will develop and refine their writing skills as they create a comprehensive report of their results.

\section*{Accessing and Using the Fibonacci Benchmarking Framework}

\begin{sloppypar} To start this laboratory assignment, you should return to the {\tt cs112S2014-share} Git repository
  and type the command {\tt git pull} in the terminal window.  Now, you should have a {\tt lab5/FibonacciBenchmark}
  directory that you can explore further.  Once again, please make sure that you can find the source code in this new
  directory and you understand why the directories in the assignment are structured the way that they are. Now, you
  should use GVim to study the source code in the {\tt build.xml} file.  As in the past assignment, you can type the
  command {\tt :make} in your GVim window and it will compile the three Java classes and save the bytecode in the
  correct subdirectories inside of the {\tt bin/} directory.  Please see the instructor if you cannot get this to work.
\end{sloppypar}

\begin{sloppypar}
As part of this assignment, we will examine the following different implementations: (i) {\tt RecursiveFibonacci} using
{\tt int}, (ii) {\tt RecursiveFibonacci} using {\tt long}, (iii) {\tt IterativeFibonacci} using {\tt int}, and (iv) {\tt
  IterativeFibonacci} using {\tt long}. The {\tt UseFibonacci} class relies on the {\tt Profiler} class is in the {\tt
  profiler.jar} file in the {\tt lib/} directory of the Git repository.  This means that the {\tt UseFibonacci} program
will not run correctly unless you have both the {\tt bin/} directory and the {\tt profiler.jar} archive inside of your
{\tt CLASSPATH} environment variable.
\end{sloppypar}

Try to execute the {\tt UseFibonacci} program for different input values from 0 to 50. You should see the output from
the computation and timing information that is related to the performance of the different algorithms. For example, try
to execute the following commands: (i) {\tt java UseFibonacci 10 all}, (ii) {\tt java UseFibonacci 10 recursive}, (iii)
{\tt java UseFibonacci 10 iterative}. 


\section*{Adding Additional Features}

If you study the source code carefully, you will see that the {\tt UseFibonacci} program currently accepts two
command-line arguments. What are they? As part of this laboratory assignment, you must extend the {\tt UseFibonacci}
class and the command-line arguments that it can accept. You should add a third command-line argument that corresponds
to one of the words {\tt int}, {\tt long}, or {\tt all}. After {\tt UseFibonacci} extracts this new command-line
argument, it should use this additional information to run a specific experiment. For example, one valid execution might
be: {\tt java UseFibonacci 10 all long}. This command line would indicate that you should only execute the methods
within {\tt RecursiveFibonacci} and {\tt IterativeFibonacci} that operate on variables with the {\tt long} primitive
data type.  You will need to add conditional logic to {\tt UseFibonacci} in order to correctly implement this additional
feature. Please see the instructor if you have questions about this task.

\section*{Conducting Experiments to Evaluate Efficiency}


\section*{Summary of the Required Deliverables}

  This assignment invites you to submit one printed version of the following deliverables: 

  \begin{enumerate} 

  \item A brief description of how recursion works in the Java programming language

  \item A short discussion of the different primitive types that are available in Java

  \item A reflective commentary on the challenges that you faced when conducting the experiments
   
  \end{enumerate}

  Along with turning in a printed version of these deliverables, you should ensure that everything is also available in
  the repository that is named according to the convention {\tt cs112S2014-<your user name>}. Please note that students
  in the class are responsible for completing and submitting their own version of this assignment.    While it is
  acceptable for members of this class to have high-level conversations, you should not share source code or full
  command lines with your classmates.  Deliverables that are nearly identical to the work of others will be taken as
  evidence of violating the \mbox{Honor Code}.  Please see the instructor if you have questions about the policies for
  this assignment.

  \end{document}
