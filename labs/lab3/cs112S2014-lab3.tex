%!TEX root=cs112S2014-lab3.tex 
% mainfile: cs112S2014-lab3.tex 

%!TEX root=cs112S2014-lab2.tex
% mainfile: cs112S2014-lab2.tex 
% Typical usage (all UPPERCASE items are optional):
%       \input 580pre
%       \begin{document}
%       \MYTITLE{Title of document, e.g., Lab 1\\Due ...}
%       \MYHEADERS{short title}{other running head, e.g., due date}
%       \PURPOSE{Description of purpose}
%       \SUMMARY{Very short overview of assignment}
%       \DETAILS{Detailed description}
%         \SUBHEAD{if needed} ...
%         \SUBHEAD{if needed} ...
%          ...
%       \HANDIN{What to hand in and how}
%       \begin{checklist}
%       \item ...
%       \end{checklist}
% There is no need to include a "\documentstyle."
% However, there should be an "\end{document}."
%
%===========================================================
\documentclass[11pt,twoside,titlepage]{article}
%%NEED TO ADD epsf!!
\usepackage{threeparttop}
\usepackage{graphicx}
\usepackage{latexsym}
\usepackage{color}
\usepackage{listings}
\usepackage{fancyvrb}
%\usepackage{pgf,pgfarrows,pgfnodes,pgfautomata,pgfheaps,pgfshade}
\usepackage{tikz}
\usepackage[normalem]{ulem}
\tikzset{
    %Define standard arrow tip
%    >=stealth',
    %Define style for boxes
    oval/.style={
           rectangle,
           rounded corners,
           draw=black, very thick,
           text width=6.5em,
           minimum height=2em,
           text centered},
    % Define arrow style
    arr/.style={
           ->,
           thick,
           shorten <=2pt,
           shorten >=2pt,}
}
\usepackage[noend]{algorithmic}
\usepackage[noend]{algorithm}
\newcommand{\bfor}{{\bf for\ }}
\newcommand{\bthen}{{\bf then\ }}
\newcommand{\bwhile}{{\bf while\ }}
\newcommand{\btrue}{{\bf true\ }}
\newcommand{\bfalse}{{\bf false\ }}
\newcommand{\bto}{{\bf to\ }}
\newcommand{\bdo}{{\bf do\ }}
\newcommand{\bif}{{\bf if\ }}
\newcommand{\belse}{{\bf else\ }}
\newcommand{\band}{{\bf and\ }}
\newcommand{\breturn}{{\bf return\ }}
\newcommand{\mod}{{\rm mod}}
\renewcommand{\algorithmiccomment}[1]{$\rhd$ #1}
\newenvironment{checklist}{\par\noindent\hspace{-.25in}{\bf Checklist:}\renewcommand{\labelitemi}{$\Box$}%
\begin{itemize}}{\end{itemize}}
\pagestyle{threepartheadings}
\usepackage{url}
\usepackage{wrapfig}
% removing the standard hyperref to avoid the horrible boxes
%\usepackage{hyperref}
\usepackage[hidelinks]{hyperref}
% added in the dtklogos for the bibtex formatting
\usepackage{dtklogos}
%=========================
% One-inch margins everywhere
%=========================
\setlength{\topmargin}{0in}
\setlength{\textheight}{8.5in}
\setlength{\oddsidemargin}{0in}
\setlength{\evensidemargin}{0in}
\setlength{\textwidth}{6.5in}
%===============================
%===============================
% Macro for document title:
%===============================
\newcommand{\MYTITLE}[1]%
   {\begin{center}
     \begin{center}
     \bf
     CMPSC 112\\Introduction to Computer Science II\\
     Spring 2014
     \medskip
     \end{center}
     \bf
     #1
     \end{center}
}
%================================
% Macro for headings:
%================================
\newcommand{\MYHEADERS}[2]%
   {\lhead{#1}
    \rhead{#2}
    %\immediate\write16{}
    %\immediate\write16{DATE OF HANDOUT?}
    %\read16 to \dateofhandout
    \def \dateofhandout {January 28, 2014}
    \lfoot{\sc Handed out on \dateofhandout}
    %\immediate\write16{}
    %\immediate\write16{HANDOUT NUMBER?}
    %\read16 to\handoutnum
    \def \handoutnum {3}
    \rfoot{Handout \handoutnum}
   }

%================================
% Macro for bold italic:
%================================
\newcommand{\bit}[1]{{\textit{\textbf{#1}}}}

%=========================
% Non-zero paragraph skips.
%=========================
\setlength{\parskip}{1ex}

%=========================
% Create various environments:
%=========================
\newcommand{\PURPOSE}{\par\noindent\hspace{-.25in}{\bf Purpose:\ }}
\newcommand{\SUMMARY}{\par\noindent\hspace{-.25in}{\bf Summary:\ }}
\newcommand{\DETAILS}{\par\noindent\hspace{-.25in}{\bf Details:\ }}
\newcommand{\HANDIN}{\par\noindent\hspace{-.25in}{\bf Hand in:\ }}
\newcommand{\SUBHEAD}[1]{\bigskip\par\noindent\hspace{-.1in}{\sc #1}\\}
%\newenvironment{CHECKLIST}{\begin{itemize}}{\end{itemize}}


\usepackage[compact]{titlesec}

\begin{document} \MYTITLE{Laboratory Assignment Three: A First Step Towards Implementing a Twitter Client}
\MYHEADERS{Laboratory Assignment Three}{Due: February 11, 2014}

\section*{Introduction}

  Twitter is a commonly used social media network.  As we refamiliarize ourselves with the fundamental characteristics
  of the object-oriented programming paradigm and continue to learn more advanced features of the Vim integrated
  development environment (IDE), we will take the first steps towards implementing a complete Twitter client. Along with
  learning how to use, enhance, and extend a Java class, we will also review the steps that you must take to perform
  console input and output. This laboratory assignment represents the first of a two-part series of assignments. 

\section*{Accessing and Understanding GatorTweet}

  To start this laboratory assignment, you should return to the {\tt cs112S2014-share} Git repository and type the
  command {\tt git pull} in the terminal window.  Now, you should have a {\tt lab3/GatorTweet/} directory that you can
  explore further.  Where is the Java source code in this new directory? Why is it stored in the directory where you
  found it? Now, you should use GVim to study the source code in the {\tt build.xml} file.  What do you think is the
  purpose of this file? For this assignment, you can type the command {\tt :make} in your GVim window and it will
  compile the {\tt Tweet.java} file and save the bytecode in the correct subdirectories inside of the {\tt bin/}
  directory. 

  At this point, you should purposefully insert an error into the {\tt Tweet.java} file.  For instance, you can open the
  file in GVim and then remove one of the semi-colons at the end of a source code line. Now, type {\tt :make} again in
  GVim.  What output do you see on the screen? GVim can provide you with a useful listing of these errors if you type
  {\tt <,q>} and then navigate around this ``Quickfix List''. When you want to zoom to a location in your Java program
  that contains an error, you can find it in the quickfix list and then press enter.  After you have returned to the
  error location, go ahead and fix the problem by adding back the semi-colon and recompiling your program. 

\section*{Improving the GatorTweet System}

  You will notice that the {\tt Tweet} class contains two instance variables and several methods.  First, you should add
  proper JavaDoc comments to all of the variables and methods in {\tt Tweet.java}.  Once you have finished adding in all
  of the necessary comments, please notice that the {\tt isValidMessage} method is not correctly implemented.  You
  should enhance this method so that it returns {\tt true} when a candidate message is valid and {\tt false} otherwise.
  This method should classify a message as valid as long as its length is greater than zero and less than or equal to
  140.

% \section*{Adding a GatorTweet Class}

Now, you should implement a {\tt GatorTweet} class that performs the following operations in {\tt main}:

\vspace*{-.1in} \begin{enumerate}
    % \itemsep 0in
    \item Constructs an {\tt ArrayList} that will only hold {\tt Tweet} objects for the valid {\tt Tweet}s.

    \item Constructs an {\tt ArrayList} that will only hold {\tt String}s for the invalid tweet messages.

    \item Uses the {\tt java.util.Scanner} call to read in consecutive lines of text from a file in the {\tt tweets/}
      subdirectory of the {\tt GatorTweet/} directory, called ``Tweets.txt'', that contains potential tweets. (It is
      worth noting that this file can contain both valid and invalid tweets).

    \item Checks to see if the current tweet message is a valid one or not.  If the message is valid, then it should
      construct a new {\tt Tweet} and store it inside of the {\tt ArrayList} for the valid instances of the {\tt Tweet}
      class.  If the message is invalid, then it should not construct a {\tt Tweet} and instead store the {\tt String}
      message in the {\tt ArrayList} for invalid tweet messages.

    \item Outputs all of the {\tt Tweet}s stored in the {\tt ArrayList} for the valid tweet messages. 

    \item Outputs all of the {\tt Tweet}s stored in the {\tt ArrayList} for the invalid tweet messages. 

  \end{enumerate}

  In the next laboratory assignment we will finish implementing a {\tt GatorTweet} program that can interact with
  Twitter. For instance, we will add features that can authenticate a user with the Twitter servers, display a user's
  Twitter timeline, and post a {\tt Tweet} to the Twitter site.

\section*{Summary of the Required Deliverables}

  This assignment invites you to submit one printed version of the following deliverables: 

  \begin{enumerate} 
    \item A description of the meaning and purpose of the provided {\tt build.xml} file.
    
    \item A screenshot demonstrating that you can display and use the quickfix list in GVim.
    
    \item The final version of your {\tt Tweet.java} and {\tt GatorTweet.java} files.
    
    \item The output associated with running {\tt GatorTweet} with a sample ``Tweets.txt'' file. 
  
  \end{enumerate}

  Along with turning in a printed version of these deliverables, you should ensure that everything is also available in
  the repository that is named according to the convention {\tt cs112S2014-<your user name>}. Please note that students
  in the class are responsible for completing and submitting their own version of this assignment.    While it is
  acceptable for members of this class to have high-level conversations, you should not share source code or full
  command lines with your classmates.  Please see the instructor if you have questions about the policies for this
  laboratory assignment.

  \end{document}
