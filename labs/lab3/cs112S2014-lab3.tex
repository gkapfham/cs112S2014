%!TEX root=cs112S2014-lab3.tex
% mainfile: cs112S2014-lab3.tex 

%!TEX root=cs112S2014-lab2.tex
% mainfile: cs112S2014-lab2.tex 
% Typical usage (all UPPERCASE items are optional):
%       \input 580pre
%       \begin{document}
%       \MYTITLE{Title of document, e.g., Lab 1\\Due ...}
%       \MYHEADERS{short title}{other running head, e.g., due date}
%       \PURPOSE{Description of purpose}
%       \SUMMARY{Very short overview of assignment}
%       \DETAILS{Detailed description}
%         \SUBHEAD{if needed} ...
%         \SUBHEAD{if needed} ...
%          ...
%       \HANDIN{What to hand in and how}
%       \begin{checklist}
%       \item ...
%       \end{checklist}
% There is no need to include a "\documentstyle."
% However, there should be an "\end{document}."
%
%===========================================================
\documentclass[11pt,twoside,titlepage]{article}
%%NEED TO ADD epsf!!
\usepackage{threeparttop}
\usepackage{graphicx}
\usepackage{latexsym}
\usepackage{color}
\usepackage{listings}
\usepackage{fancyvrb}
%\usepackage{pgf,pgfarrows,pgfnodes,pgfautomata,pgfheaps,pgfshade}
\usepackage{tikz}
\usepackage[normalem]{ulem}
\tikzset{
    %Define standard arrow tip
%    >=stealth',
    %Define style for boxes
    oval/.style={
           rectangle,
           rounded corners,
           draw=black, very thick,
           text width=6.5em,
           minimum height=2em,
           text centered},
    % Define arrow style
    arr/.style={
           ->,
           thick,
           shorten <=2pt,
           shorten >=2pt,}
}
\usepackage[noend]{algorithmic}
\usepackage[noend]{algorithm}
\newcommand{\bfor}{{\bf for\ }}
\newcommand{\bthen}{{\bf then\ }}
\newcommand{\bwhile}{{\bf while\ }}
\newcommand{\btrue}{{\bf true\ }}
\newcommand{\bfalse}{{\bf false\ }}
\newcommand{\bto}{{\bf to\ }}
\newcommand{\bdo}{{\bf do\ }}
\newcommand{\bif}{{\bf if\ }}
\newcommand{\belse}{{\bf else\ }}
\newcommand{\band}{{\bf and\ }}
\newcommand{\breturn}{{\bf return\ }}
\newcommand{\mod}{{\rm mod}}
\renewcommand{\algorithmiccomment}[1]{$\rhd$ #1}
\newenvironment{checklist}{\par\noindent\hspace{-.25in}{\bf Checklist:}\renewcommand{\labelitemi}{$\Box$}%
\begin{itemize}}{\end{itemize}}
\pagestyle{threepartheadings}
\usepackage{url}
\usepackage{wrapfig}
% removing the standard hyperref to avoid the horrible boxes
%\usepackage{hyperref}
\usepackage[hidelinks]{hyperref}
% added in the dtklogos for the bibtex formatting
\usepackage{dtklogos}
%=========================
% One-inch margins everywhere
%=========================
\setlength{\topmargin}{0in}
\setlength{\textheight}{8.5in}
\setlength{\oddsidemargin}{0in}
\setlength{\evensidemargin}{0in}
\setlength{\textwidth}{6.5in}
%===============================
%===============================
% Macro for document title:
%===============================
\newcommand{\MYTITLE}[1]%
   {\begin{center}
     \begin{center}
     \bf
     CMPSC 112\\Introduction to Computer Science II\\
     Spring 2014
     \medskip
     \end{center}
     \bf
     #1
     \end{center}
}
%================================
% Macro for headings:
%================================
\newcommand{\MYHEADERS}[2]%
   {\lhead{#1}
    \rhead{#2}
    %\immediate\write16{}
    %\immediate\write16{DATE OF HANDOUT?}
    %\read16 to \dateofhandout
    \def \dateofhandout {January 28, 2014}
    \lfoot{\sc Handed out on \dateofhandout}
    %\immediate\write16{}
    %\immediate\write16{HANDOUT NUMBER?}
    %\read16 to\handoutnum
    \def \handoutnum {3}
    \rfoot{Handout \handoutnum}
   }

%================================
% Macro for bold italic:
%================================
\newcommand{\bit}[1]{{\textit{\textbf{#1}}}}

%=========================
% Non-zero paragraph skips.
%=========================
\setlength{\parskip}{1ex}

%=========================
% Create various environments:
%=========================
\newcommand{\PURPOSE}{\par\noindent\hspace{-.25in}{\bf Purpose:\ }}
\newcommand{\SUMMARY}{\par\noindent\hspace{-.25in}{\bf Summary:\ }}
\newcommand{\DETAILS}{\par\noindent\hspace{-.25in}{\bf Details:\ }}
\newcommand{\HANDIN}{\par\noindent\hspace{-.25in}{\bf Hand in:\ }}
\newcommand{\SUBHEAD}[1]{\bigskip\par\noindent\hspace{-.1in}{\sc #1}\\}
%\newenvironment{CHECKLIST}{\begin{itemize}}{\end{itemize}}


\usepackage[compact]{titlesec}

\begin{document}
\MYTITLE{Laboratory Assignment Three: A First Step Towards Implementing a Twitter Client}
\MYHEADERS{Laboratory Assignment Three}{Due: February 11, 2014}

\section*{Introduction}

Twitter is a commonly used social media network.  As we refamiliarize ourselves with the fundamental characteristics of the
object-oriented programming paradigm and continue to learn more advanced features of the Vim integrated development environment
(IDE), we will take the first steps towards implementing a complete Twitter client. Along with learning how to use, enhance, and
extend a Java class, we will also review the steps that you must take to perform console input and output. This laboratory
assignment represents the first of a two-part series of assignments. 

\section*{Accessing and Understanding GatorTweet}

To start this laboratory assignment, you should return to the {\tt cs112S2014-share} Git repository and type the command {\tt git
pull} in the terminal window.  Now, you should have a {\tt lab3/GatorTweet/} directory that you can explore further.  Where is
the Java source code in this new directory? Why is it stored in the directory where you found it? Now, you should use GVim to
study the source code in the {\tt build.xml} file.  What do you think is the purpose of this file? For this assignment, you can
type the command {\tt :make} in your GVim window and it will compile the {\tt Tweet.java} class and save the bytecode in the
correct subdirectories inside of the {\tt bin} directory. 

At this point, you should purposefully insert an error into the {\tt Tweet.java} class.  For instance, you can open the file in
GVim and then remove one of the semi-colons at the end of a source code line. Now, type {\tt :make} again in GVim.  What output do
you see on the screen? GVim can provide you a useful listing of these errors if you type {\tt <,q>} and then navigate around this
``Quickfix List''. When you want to zoom to a location in your Java program that contains an error, you can find it in the
quickfix list and then press enter.  After you have returned to the error location, go ahead and fix the problem by adding back
the semi-colon and recompiling your program. 

\section*{Summary of the Required Deliverables}

  This assignment invites you to submit one printed version of a tutorial that contains:

  \begin{enumerate}
    \item A commentary on how Vim uses runtime configuration files
    \item A full-featured description of the basic features associated with the Vim text editor
    \item A complete introduction to the use of the Vim aforementioned plugins 
  \end{enumerate}

  Along with turning in a printed version of your tutorial, you should ensure that your document is also available in the repository
  that is named according to the convention {\tt cs112S2014-<your user name>}. Please note that students in the class are
  responsible for completing and submitting their own version of this assignment.  However, you also will be assigned to work in a
  team that is tasked with ensuring that all of its members are able to complete each step of the assignment.  Team members should
  make themselves available to each other to answer questions and resolve any problems that develop during the laboratory session.
  While it is acceptable for members of a team to have high-level conversations, you should not share source code or full command
  lines with your team members. To ensure that you can communicate effectively, members of each team should sit next to each other
  in the room.  Please see the instructor if you have questions about this policy.

  \end{document}
