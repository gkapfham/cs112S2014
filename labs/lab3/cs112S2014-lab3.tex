%!TEX root=cs112S2014-lab3.tex
% mainfile: cs112S2014-lab3.tex 

\input{labspre.tex}

\usepackage[compact]{titlesec}

\begin{document}
\MYTITLE{Laboratory Assignment Three: A First Step Towards Implementing a Twitter Client}
\MYHEADERS{Laboratory Assignment Three}{Due: February 11, 2014}

\section*{Introduction}

\section*{Summary of the Required Deliverables}

  This assignment invites you to submit one printed version of a tutorial that contains:

  \begin{enumerate}
    \item A commentary on how Vim uses runtime configuration files
    \item A full-featured description of the basic features associated with the Vim text editor
    \item A complete introduction to the use of the Vim aforementioned plugins 
  \end{enumerate}

  Along with turning in a printed version of your tutorial, you should ensure that your document is also available in the repository
  that is named according to the convention {\tt cs112S2014-<your user name>}. Please note that students in the class are
  responsible for completing and submitting their own version of this assignment.  However, you also will be assigned to work in a
  team that is tasked with ensuring that all of its members are able to complete each step of the assignment.  Team members should
  make themselves available to each other to answer questions and resolve any problems that develop during the laboratory session.
  While it is acceptable for members of a team to have high-level conversations, you should not share source code or full command
  lines with your team members. To ensure that you can communicate effectively, members of each team should sit next to each other
  in the room.  Please see the instructor if you have questions about this policy.

  \end{document}
