%!TEX root=cs112S2014-fp.tex 
% mainfile: cs112S2014-fp.tex 

\input{labspre.tex}

\usepackage[compact]{titlesec}

\begin{document} \MYTITLE{Final Project: Automated Analysis of Todo.txt Files}
\MYHEADERS{Final Project}{Due: May 6, 2014 at 5:00 pm}

\section*{Introduction}

Todo.txt is a file format and a suite of tools that supports the management of todo lists and the completion of project
tasks. Originally developed by Gina Trapani, Todo.txt currently includes an ecosystem of tools such as mobile apps and
desktop applications, that enable individuals and teams to manage projects. In this final project, you and your team
members will use Todo.txt to track your progress as you build a tool that can automatically analyze Todo.txt files.

Current Todo.txt tools (e.g., todo.txt-cli, todo.txt-vim, and the Todo.txt Android app) enable a user to input and mark
as completed a wide variety of tasks.  Yet, right now there is no tool that can automatically analyze an existing
Todo.txt file and determine how the time of an individual or a team was spent. You and your team members will implement
a complete tool for the automated analysis of Todo.txt files.  For instance, your tool should be able to determine how
many high-priority tasks you recently completed.  In addition, your tool should be able to ascertain how many tasks were
completed in the different contexts and projects represented in a specified Todo.txt file. 

As you specify, design, implement, test, debug, demonstrate, and release your Todo.txt analysis system, you and your
team members will develop your knowledge and skills in computer science as you learn more about advanced command-line
argument management, the reading and parsing of an input file, and the analysis of frequencies with the hashtable
abstract data type. You will also learn more about practice of real-world software development as you work in teams and
manage the schedule of your project. Finally, you will hone your writing and speaking skills and you document and
present your working tool for the analysis of Todo.txt files.

\section*{Learning About the Todo.txt Format}

\section*{Summary of the Required Deliverables}

  This assignment invites you to submit a signed and printed version of the following deliverables: 

  \begin{enumerate} 
  \itemsep0pt

  \item A complete description of a problem that can be solved by using the queue abstract data type

  \item The properly commented version of {\tt JosephusSolver.java} and any other Java files you create

  \item The output from five separate runs of {\tt JosephusSolver}, demonstrating it's correctness 

  \item A written report that provides a response to all of the questions posed in this assignment

  \item A justified statement of the worst-case time complexity for the {\tt Josephus} method

  \item A written analysis of the {\tt JosephusSolver}'s efficiency when it is run in different configurations 

  \item A reflective commentary on the challenges that you faced when completing this assignment 
   
  \end{enumerate}

  Along with turning in a printed version of these deliverables, you should ensure that everything is also available in
  the repository that is named according to the convention {\tt cs112S2014-<your user name>}. Please note that students
  in the class are responsible for completing and submitting their own version of this assignment.    While it is
  acceptable for members of this class to have high-level conversations, you should not share source code or full
  command lines with your classmates.  Deliverables that are nearly identical to the work of others will be taken as
  evidence of violating the \mbox{Honor Code}.  Please see the instructor if you have questions about the policies for
  this assignment.

  \end{document}
