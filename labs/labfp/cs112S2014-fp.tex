%!TEX root=cs112S2014-fp.tex 
% mainfile: cs112S2014-fp.tex 

%!TEX root=cs112S2014-lab2.tex
% mainfile: cs112S2014-lab2.tex 
% Typical usage (all UPPERCASE items are optional):
%       \input 580pre
%       \begin{document}
%       \MYTITLE{Title of document, e.g., Lab 1\\Due ...}
%       \MYHEADERS{short title}{other running head, e.g., due date}
%       \PURPOSE{Description of purpose}
%       \SUMMARY{Very short overview of assignment}
%       \DETAILS{Detailed description}
%         \SUBHEAD{if needed} ...
%         \SUBHEAD{if needed} ...
%          ...
%       \HANDIN{What to hand in and how}
%       \begin{checklist}
%       \item ...
%       \end{checklist}
% There is no need to include a "\documentstyle."
% However, there should be an "\end{document}."
%
%===========================================================
\documentclass[11pt,twoside,titlepage]{article}
%%NEED TO ADD epsf!!
\usepackage{threeparttop}
\usepackage{graphicx}
\usepackage{latexsym}
\usepackage{color}
\usepackage{listings}
\usepackage{fancyvrb}
%\usepackage{pgf,pgfarrows,pgfnodes,pgfautomata,pgfheaps,pgfshade}
\usepackage{tikz}
\usepackage[normalem]{ulem}
\tikzset{
    %Define standard arrow tip
%    >=stealth',
    %Define style for boxes
    oval/.style={
           rectangle,
           rounded corners,
           draw=black, very thick,
           text width=6.5em,
           minimum height=2em,
           text centered},
    % Define arrow style
    arr/.style={
           ->,
           thick,
           shorten <=2pt,
           shorten >=2pt,}
}
\usepackage[noend]{algorithmic}
\usepackage[noend]{algorithm}
\newcommand{\bfor}{{\bf for\ }}
\newcommand{\bthen}{{\bf then\ }}
\newcommand{\bwhile}{{\bf while\ }}
\newcommand{\btrue}{{\bf true\ }}
\newcommand{\bfalse}{{\bf false\ }}
\newcommand{\bto}{{\bf to\ }}
\newcommand{\bdo}{{\bf do\ }}
\newcommand{\bif}{{\bf if\ }}
\newcommand{\belse}{{\bf else\ }}
\newcommand{\band}{{\bf and\ }}
\newcommand{\breturn}{{\bf return\ }}
\newcommand{\mod}{{\rm mod}}
\renewcommand{\algorithmiccomment}[1]{$\rhd$ #1}
\newenvironment{checklist}{\par\noindent\hspace{-.25in}{\bf Checklist:}\renewcommand{\labelitemi}{$\Box$}%
\begin{itemize}}{\end{itemize}}
\pagestyle{threepartheadings}
\usepackage{url}
\usepackage{wrapfig}
% removing the standard hyperref to avoid the horrible boxes
%\usepackage{hyperref}
\usepackage[hidelinks]{hyperref}
% added in the dtklogos for the bibtex formatting
\usepackage{dtklogos}
%=========================
% One-inch margins everywhere
%=========================
\setlength{\topmargin}{0in}
\setlength{\textheight}{8.5in}
\setlength{\oddsidemargin}{0in}
\setlength{\evensidemargin}{0in}
\setlength{\textwidth}{6.5in}
%===============================
%===============================
% Macro for document title:
%===============================
\newcommand{\MYTITLE}[1]%
   {\begin{center}
     \begin{center}
     \bf
     CMPSC 112\\Introduction to Computer Science II\\
     Spring 2014
     \medskip
     \end{center}
     \bf
     #1
     \end{center}
}
%================================
% Macro for headings:
%================================
\newcommand{\MYHEADERS}[2]%
   {\lhead{#1}
    \rhead{#2}
    %\immediate\write16{}
    %\immediate\write16{DATE OF HANDOUT?}
    %\read16 to \dateofhandout
    \def \dateofhandout {January 28, 2014}
    \lfoot{\sc Handed out on \dateofhandout}
    %\immediate\write16{}
    %\immediate\write16{HANDOUT NUMBER?}
    %\read16 to\handoutnum
    \def \handoutnum {3}
    \rfoot{Handout \handoutnum}
   }

%================================
% Macro for bold italic:
%================================
\newcommand{\bit}[1]{{\textit{\textbf{#1}}}}

%=========================
% Non-zero paragraph skips.
%=========================
\setlength{\parskip}{1ex}

%=========================
% Create various environments:
%=========================
\newcommand{\PURPOSE}{\par\noindent\hspace{-.25in}{\bf Purpose:\ }}
\newcommand{\SUMMARY}{\par\noindent\hspace{-.25in}{\bf Summary:\ }}
\newcommand{\DETAILS}{\par\noindent\hspace{-.25in}{\bf Details:\ }}
\newcommand{\HANDIN}{\par\noindent\hspace{-.25in}{\bf Hand in:\ }}
\newcommand{\SUBHEAD}[1]{\bigskip\par\noindent\hspace{-.1in}{\sc #1}\\}
%\newenvironment{CHECKLIST}{\begin{itemize}}{\end{itemize}}


\usepackage[compact]{titlesec}

\begin{document} \MYTITLE{Final Project: Automated Analysis of Todo.txt Files}
\MYHEADERS{Final Project}{Due: May 6, 2014 at 12:00 noon}

\section*{Introduction}

Todo.txt is a file format and a suite of tools that supports the management of todo lists and the completion of project
tasks. Originally developed by Gina Trapani, Todo.txt currently includes an ecosystem of tools such as mobile apps and
desktop applications, that enable individuals and teams to manage projects. In this final project, you and your team
members will use Todo.txt to track your progress as you build a tool that can automatically analyze Todo.txt files.

Current Todo.txt tools (e.g., todo.txt-cli, todo.txt-vim, and the Todo.txt Android app) enable a user to input and mark
as completed a wide variety of tasks.  Yet, right now there is no tool that can automatically analyze an existing
Todo.txt file and determine how the time of an individual or a team was spent. You and your team members will implement
a complete tool for the automated analysis of Todo.txt files.  For instance, your tool should be able to determine how
many high-priority tasks you recently completed.  In addition, your tool should be able to ascertain how many tasks were
completed in the different contexts and projects represented in a specified Todo.txt file. 

As you specify, design, implement, test, debug, demonstrate, and release your Todo.txt analysis system, you and your
team members will develop your knowledge and skills in computer science as you learn more about advanced command-line
argument management, the reading and parsing of an input file, and the analysis of frequencies with the hashtable
abstract data type. You will also learn more about practice of real-world software development as you work in teams and
manage the schedule of your project. Finally, you will hone your writing and speaking skills as you document, present,
and release your working tool for the analysis of Todo.txt files.

\section*{Understanding and Using Todo.txt}

Before you start to specify, design, implement, and test your system, you need to learn more about the Todo.txt
ecosystem that is described at \url{http://todotxt.com/}. You and your team members should study and discuss this site
so that you understand the philosophy behind the Todo.txt system. To gain more experience with using the Todo.txt
system, each of your team members should install and use the todo.txt-vim plugin that is available for download from the
\url{https://github.com/freitass/todo.txt-vim} Web site. To get this plugin fully working, you will need to add a {\tt
  Bundle} command to your {\tt .vimrc} file and then run the {\tt BundleInstall} command in a separate instance of Vim.
To ensure that Vim always enters into todo-mode when you edit a Todo.txt file, you should add the following line to your
{\tt .vimrc}:

{\tt autocmd BufNewFile,BufRead [Tt]odo.txt set filetype=todo}

Once you have properly configured Vim, you are ready to investigate the use of a Todo.txt file. After creating a
directory in your account called {\tt todo}, you should edit a file called Todo.txt.  Before you start to add content to
this file, you need to learn more about the Todo.txt format, which is described at
\url{https://github.com/ginatrapani/todo.txt-cli/wiki/The-Todo.txt-Format}. Students who would like to study an example
of an existing Todo.txt file can view the one available at \url{http://todotxt.com/todo.txt}. Adding a new task to your
todo list is as simple as starting to type in the Vim text editor.  Moreover, you can mark a task as completed by using
the command {\tt <leader>D} and you can sort all of the tasks in your Todo.txt file by typing {\tt <leader>s}. Before
you finish this phase of the final project, please make sure that all of your team members understand how to use the
context and project annotations supported by the Todo.txt format.

\section*{Advanced Command-Line Argument Handling}

In past laboratory assignments, you often handled command-line arguments by writing code that would inspect the values
in the {\tt args[]} that is a parameter to the {\tt main} method.  However, the programs that we have used during our
class and laboratory sessions handled the command-line arguments in a brittle fashion: they required the command-line
arguments to appear in a fixed order and they did not allow for the specification of arguments in a fashion similar to
that used by most programs run in the Linux terminal window. Thankfully, there is a third-party Java library, called
JCommander, that makes it easier to perform advanced argument management. 

Since we want our Todo.txt analysis tool to have a full-featured and easy-to-use command-line argument system, we will
use JCommander for this final project.  You and your team members should learn how to use JCommander by visiting the
following Web sites: \url{http://jcommander.org/} and \url{https://github.com/cbeust/jcommander}. You can find a current
version of the JCommander JAR file by running the {\tt git pull} command in the {\tt cs112S2014-share/} repository.
After studying JCommander's user documentation, discussing the library with your team members, and resolving any
questions by talking with the course instructor, your team should consider trying a simple example with JCommander. Can
you display a help menu and recognize arguments?

\section*{Reading and Parsing Todo.txt Entries}

\section*{Frequency Analysis with Hashtables}

\section*{Todo.txt Analysis Tasks}

\section*{Summary of the Required Deliverables}

  This assignment invites you to submit a signed and printed version of the following deliverables: 

  \begin{enumerate} 
  \itemsep0pt

  \item A complete description of a problem that can be solved by using the queue abstract data type

  \item The properly commented version of {\tt JosephusSolver.java} and any other Java files you create

  \item The output from five separate runs of {\tt JosephusSolver}, demonstrating it's correctness 

  \item A written report that provides a response to all of the questions posed in this assignment

  \item A justified statement of the worst-case time complexity for the {\tt Josephus} method

  \item A written analysis of the {\tt JosephusSolver}'s efficiency when it is run in different configurations 

  \item A reflective commentary on the challenges that you faced when completing this assignment 
   
  \end{enumerate}

  Along with turning in a printed version of these deliverables, you should ensure that everything is also available in
  the repository that is named according to the convention {\tt cs112S2014-<your user name>}. Please note that students
  in the class are responsible for completing and submitting their own version of this assignment.    While it is
  acceptable for members of this class to have high-level conversations, you should not share source code or full
  command lines with your classmates.  Deliverables that are nearly identical to the work of others will be taken as
  evidence of violating the \mbox{Honor Code}.  Please see the instructor if you have questions about the policies for
  this assignment.

  \end{document}
